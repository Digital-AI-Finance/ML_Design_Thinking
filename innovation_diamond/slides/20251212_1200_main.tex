\documentclass[8pt,aspectratio=169]{beamer}
\usetheme{Madrid}

% ============================================================
% PACKAGES
% ============================================================
\usepackage{graphicx}
\usepackage{booktabs}
\usepackage{adjustbox}
\usepackage{multicol}
\usepackage{amsmath}

% ============================================================
% DIAMOND COLOR SCHEME
% ============================================================
\definecolor{challenge}{RGB}{148,103,189}
\definecolor{explore}{RGB}{52,152,219}
\definecolor{generate}{RGB}{46,204,113}
\definecolor{peak}{RGB}{241,196,15}
\definecolor{filter}{RGB}{230,126,34}
\definecolor{refine}{RGB}{231,76,60}
\definecolor{strategy}{RGB}{192,57,43}
\definecolor{mlpurple}{RGB}{51,51,178}
\definecolor{mllavender}{RGB}{173,173,224}
\definecolor{mllavender2}{RGB}{193,193,232}
\definecolor{mllavender3}{RGB}{204,204,235}
\definecolor{darkgray}{RGB}{64,64,64}

% ============================================================
% THEME CONFIGURATION
% ============================================================
\setbeamercolor{structure}{fg=mlpurple}
\setbeamercolor{frametitle}{fg=mlpurple,bg=mllavender3}
\setbeamercolor{block title}{fg=mlpurple,bg=mllavender2}
\setbeamercolor{block body}{bg=mllavender3!30}
\setbeamertemplate{navigation symbols}{}
\setbeamertemplate{itemize item}{\textbullet}
\setbeamertemplate{itemize subitem}{--}
\setbeamersize{text margin left=5mm,text margin right=5mm}

% ============================================================
% CUSTOM COMMANDS
% ============================================================
\newcommand{\bottomnote}[1]{%
    \vfill
    \textcolor{mllavender2}{\rule{\textwidth}{0.4pt}}
    \par\vspace{2pt}
    {\footnotesize\textcolor{mllavender2}{#1}}
}

\newcommand{\stagecolor}[2]{\textcolor{#1}{\textbf{#2}}}
\newcommand{\pitfall}[1]{\vspace{0.3em}\textcolor{refine}{\textbf{Pitfall:}} \textit{#1}}
\newcommand{\sigeq}[1]{\vspace{0.3em}\colorbox{mllavender3!50}{$\displaystyle #1$}}

% ============================================================
% DOCUMENT INFO
% ============================================================
\title{ML-Powered Innovation}
\subtitle{From Challenge to Strategy: The Innovation Diamond}
\author{Machine Learning for Smarter Innovation}
\date{BSc Course Capstone}

\begin{document}

% ============================================================
% PART 1: OPENING (4 slides)
% ============================================================

% Slide 1: Title
\begin{frame}[plain]
\titlepage
\end{frame}

% Slide 2: Course Roadmap
\begin{frame}{Course Roadmap: 14 Topics, One Framework}
\begin{columns}[T]
\begin{column}{0.48\textwidth}
\textbf{Foundations}
\begin{itemize}
\item ML Foundations
\item Supervised Learning
\item Unsupervised Learning
\item Neural Networks
\end{itemize}

\textbf{Core Techniques}
\begin{itemize}
\item Clustering
\item Classification
\item NLP \& Sentiment
\item Topic Modeling
\end{itemize}
\end{column}

\begin{column}{0.48\textwidth}
\textbf{Advanced Applications}
\begin{itemize}
\item Generative AI
\item Structured Output
\item Validation \& Metrics
\item A/B Testing
\end{itemize}

\textbf{Specialized}
\begin{itemize}
\item Responsible AI
\item Finance Applications
\end{itemize}
\end{column}
\end{columns}

\vspace{0.5em}
\centering
\textcolor{mlpurple}{\textbf{All 14 topics connect through the Innovation Diamond}}

\bottomnote{Each ML technique serves a specific purpose in the innovation journey from challenge to strategy}
\end{frame}

% Slide 3: Innovation Diamond Overview
\begin{frame}{The Innovation Diamond: From Challenge to Strategy}
\begin{center}
\includegraphics[width=0.55\textwidth]{../charts/01_diamond_overview/chart.pdf}
\end{center}
\bottomnote{ML enables both creative expansion and strategic focus in innovation}
\end{frame}

% Slide 4: ESG Challenge Introduction
\begin{frame}{The ESG Challenge: Our Running Example}
\begin{columns}[T]
\begin{column}{0.55\textwidth}
\textbf{The Innovation Challenge:}

\vspace{0.5em}
\textit{``How can we create an investment portfolio that maximizes returns while ensuring genuine environmental and social impact?''}

\vspace{1em}
\textbf{Why This Matters:}
\begin{itemize}
\item \$35 trillion in ESG assets globally
\item Greenwashing concerns abound
\item Need rigorous, data-driven approach
\end{itemize}

\vspace{0.5em}
\textbf{Our Journey:} 1 challenge $\rightarrow$ 5,000 possibilities $\rightarrow$ 5 strategies
\end{column}

\begin{column}{0.43\textwidth}
\centering
\includegraphics[width=0.9\textwidth]{../charts/04_esg_dimensions/chart.pdf}
\end{column}
\end{columns}

\bottomnote{This challenge will guide us through all 14 ML topics in the Innovation Diamond}
\end{frame}

% ============================================================
% PART 2: DIVERGENT PHASE (12 slides)
% ============================================================

% Slide 5: ML Foundations
\begin{frame}{Stage 1: \stagecolor{challenge}{Challenge} -- ML Foundations}
\begin{columns}[T]
\begin{column}{0.55\textwidth}
\textbf{The Starting Point (1)}
\begin{itemize}
\item Define the problem clearly
\item Understand stakeholder needs
\item Set measurable success criteria
\end{itemize}

\textbf{Signature Equation -- Loss Function:}

\sigeq{L(\theta) = \frac{1}{n}\sum_{i=1}^{n}\ell(y_i, f(x_i;\theta))}

\vspace{0.5em}
\textbf{ESG Application:}

Minimize prediction error for ESG impact vs. returns tradeoff

\pitfall{Starting too broad or too narrow}
\end{column}

\begin{column}{0.43\textwidth}
\centering
\includegraphics[width=0.9\textwidth]{../charts/01_diamond_overview/chart.pdf}
\end{column}
\end{columns}

\bottomnote{ML Foundations provides the vocabulary and framework for framing innovation challenges}
\end{frame}

% Slide 6: Unsupervised Learning
\begin{frame}{Stage 2: \stagecolor{explore}{Exploration} -- Unsupervised Learning}
\begin{columns}[T]
\begin{column}{0.55\textwidth}
\textbf{Exploring the Space (10 dimensions)}
\begin{itemize}
\item Identify relevant dimensions
\item Find hidden structure in data
\item No predefined labels needed
\end{itemize}

\textbf{Signature Equation -- K-Means:}

\sigeq{\underset{C}{\text{argmin}} \sum_{k}\sum_{x \in C_k}\|x - \mu_k\|^2}

\vspace{0.5em}
\textbf{ESG Application:}

10 dimensions discovered: Carbon, Water, Labor, Board diversity, Supply chain, Community, Waste, Energy, Privacy, Anti-corruption

\pitfall{Ignoring non-obvious dimensions}
\end{column}

\begin{column}{0.43\textwidth}
\centering
\includegraphics[width=0.9\textwidth]{../charts/04_esg_dimensions/chart.pdf}
\end{column}
\end{columns}

\bottomnote{Unsupervised Learning reveals hidden structure without requiring labeled examples}
\end{frame}

% Slide 7: Supervised Learning
\begin{frame}{Stage 3: \stagecolor{explore}{Discovery} -- Supervised Learning}
\begin{columns}[T]
\begin{column}{0.55\textwidth}
\textbf{Feature Engineering (100 features)}
\begin{itemize}
\item Transform raw data into features
\item Engineer domain-specific metrics
\item Validate feature relevance
\end{itemize}

\textbf{Signature Equation -- Linear Prediction:}

\sigeq{\hat{y} = \sum_{j=1}^{p}\beta_j x_j + \epsilon}

\vspace{0.5em}
\textbf{ESG Application:}

100 features from sustainability reports, news sentiment, social media, regulatory filings

\pitfall{Creating features without domain knowledge}
\end{column}

\begin{column}{0.43\textwidth}
\centering
\includegraphics[width=0.9\textwidth]{../charts/05_feature_extraction/chart.pdf}
\end{column}
\end{columns}

\bottomnote{Supervised Learning teaches which features actually predict outcomes}
\end{frame}

% Slide 8: Neural Networks
\begin{frame}{Cross-Cutting: Neural Networks}
\begin{columns}[T]
\begin{column}{0.55\textwidth}
\textbf{Deep Learning Power}
\begin{itemize}
\item Learn complex patterns
\item Automatic feature extraction
\item Transfer learning capability
\end{itemize}

\textbf{Signature Equation -- Forward Propagation:}

\sigeq{a^{(l)} = \sigma(W^{(l)}a^{(l-1)} + b^{(l)})}

\vspace{0.5em}
\textbf{ESG Application:}

Deep networks process unstructured ESG reports, extract sentiment from news, identify patterns in complex datasets

\pitfall{Black-box models reduce interpretability}
\end{column}

\begin{column}{0.43\textwidth}
\centering
\includegraphics[width=0.95\textwidth]{../charts/16_neural_network_arch/chart.pdf}
\end{column}
\end{columns}

\bottomnote{Neural Networks provide powerful pattern recognition across the entire innovation journey}
\end{frame}

% Slide 9: Generative AI
\begin{frame}{Stage 4: \stagecolor{generate}{Generation} -- Generative AI}
\begin{columns}[T]
\begin{column}{0.55\textwidth}
\textbf{Creative Expansion (1,000 ideas)}
\begin{itemize}
\item Generate diverse possibilities
\item Explore unconventional combinations
\item Push beyond obvious solutions
\end{itemize}

\textbf{Signature Equation -- Generative Model:}

\sigeq{P(x) = \int P(x|z)P(z)dz}

\vspace{0.5em}
\textbf{ESG Application:}

LLMs generate 1,000 investment thesis variations like ``Invest in circular economy leaders''

\pitfall{Generating ideas without quality filters}
\end{column}

\begin{column}{0.43\textwidth}
\centering
\includegraphics[width=0.9\textwidth]{../charts/06_idea_generation/chart.pdf}
\end{column}
\end{columns}

\bottomnote{Generative AI expands the solution space beyond human capacity}
\end{frame}

% Slide 10: Topic Modeling
\begin{frame}{Stage 4b: \stagecolor{generate}{Generation} -- Topic Modeling}
\begin{columns}[T]
\begin{column}{0.55\textwidth}
\textbf{Discovering Themes}
\begin{itemize}
\item Extract topics from documents
\item Identify latent themes
\item Organize unstructured content
\end{itemize}

\textbf{Signature Equation -- LDA:}

\sigeq{P(w|d) = \sum_{t}P(w|t)P(t|d)}

\vspace{0.5em}
\textbf{ESG Application:}

LDA discovers themes in sustainability reports: ``climate action'', ``diversity initiatives'', ``governance reforms''

\pitfall{Over-interpreting topic labels}
\end{column}

\begin{column}{0.43\textwidth}
\centering
\textbf{Topic Distribution Example}

\vspace{0.5em}
\begin{tabular}{lc}
\toprule
\textbf{Topic} & \textbf{Weight} \\
\midrule
Climate Action & 0.35 \\
Supply Chain & 0.25 \\
Governance & 0.20 \\
Social Impact & 0.15 \\
Other & 0.05 \\
\bottomrule
\end{tabular}
\end{column}
\end{columns}

\bottomnote{Topic Modeling reveals hidden themes in document collections}
\end{frame}

% Slide 11: NLP & Sentiment
\begin{frame}{Stage 5: \stagecolor{peak}{Peak} -- NLP \& Sentiment Analysis}
\begin{columns}[T]
\begin{column}{0.55\textwidth}
\textbf{Maximum Expansion (5,000 ideas)}
\begin{itemize}
\item Process massive text data
\item Extract sentiment signals
\item Combine all generated content
\end{itemize}

\textbf{Signature Equation -- Language Model:}

\sigeq{P(w_t|w_{t-k},...,w_{t-1}) = \text{softmax}(W \cdot h_t)}

\vspace{0.5em}
\textbf{ESG Application:}

5,000 potential criteria combining company targets (500+), weighting schemes (100+), sectors (50+), risk parameters (20+)

\pitfall{Analysis paralysis at peak}
\end{column}

\begin{column}{0.43\textwidth}
\centering
\includegraphics[width=0.9\textwidth]{../charts/07_peak_pool/chart.pdf}
\end{column}
\end{columns}

\bottomnote{NLP and Sentiment Analysis help process and understand massive text-based idea pools}
\end{frame}

% Slide 12: Divergent Summary
\begin{frame}{Divergent Phase Summary: Techniques That Expand}
\begin{center}
\begin{tabular}{lccc}
\toprule
\textbf{Stage} & \textbf{Count} & \textbf{ML Technique} & \textbf{Course Topic} \\
\midrule
\textcolor{challenge}{Challenge} & 1 & Problem Framing & ML Foundations \\
\textcolor{explore}{Exploration} & 10 & Data Mining & Unsupervised Learning \\
\textcolor{explore}{Discovery} & 100 & Feature Engineering & Supervised Learning \\
\textcolor{generate}{Generation} & 1,000 & Generative Algorithms & Generative AI, Topic Modeling \\
\textcolor{peak}{Peak} & 5,000 & NLP Analysis & NLP \& Sentiment \\
\bottomrule
\end{tabular}
\end{center}

\vspace{0.5em}
\textbf{Key Insight:} Each technique serves a specific expansion purpose

\begin{itemize}
\item \textbf{Unsupervised} learning finds structure without labels
\item \textbf{Generative AI} creates new possibilities
\item \textbf{NLP} processes human-generated content at scale
\end{itemize}

\bottomnote{Divergent thinking requires ML techniques that expand rather than constrain}
\end{frame}

% Slide 13: Visualizing Expansion
\begin{frame}{Visualizing Expansion: 1 to 5,000}
\begin{center}
\includegraphics[width=0.7\textwidth]{../charts/02_divergent_process/chart.pdf}
\end{center}
\bottomnote{The divergent phase systematically expands from a single challenge to thousands of possibilities}
\end{frame}

% ============================================================
% PART 3: TRANSITION (2 slides)
% ============================================================

% Slide 14: Transition
\begin{frame}{Transition: Now We Must Focus}
\begin{center}
\Large
\textcolor{peak}{\textbf{5,000 ideas}}

\vspace{1em}
$\downarrow$

\vspace{1em}
\textit{``Having many options is valuable only if you can choose wisely.''}

\vspace{1em}
$\downarrow$

\vspace{1em}
\textcolor{strategy}{\textbf{5 strategies}}
\end{center}

\vspace{1em}
The convergent phase applies ML to systematically filter, pattern-match, validate, and select.

\bottomnote{Innovation requires both expansion and focus -- now we apply convergent ML techniques}
\end{frame}

% ============================================================
% PART 4: CONVERGENT PHASE (12 slides)
% ============================================================

% Slide 15: Clustering
\begin{frame}{Stage 6: \stagecolor{filter}{Extraction} -- Clustering}
\begin{columns}[T]
\begin{column}{0.55\textwidth}
\textbf{Initial Filtering (2,000)}
\begin{itemize}
\item Group similar ideas
\item Remove duplicates and noise
\item Identify natural clusters
\end{itemize}

\textbf{Signature Equation -- Silhouette Score:}

\sigeq{\text{Silhouette}(i) = \frac{b(i) - a(i)}{\max(a(i), b(i))}}

\vspace{0.5em}
\textbf{ESG Application:}

8 distinct clusters: Best-in-class, Exclusionary, Impact-first, ESG momentum, etc.

\pitfall{Forcing clusters that don't exist}
\end{column}

\begin{column}{0.43\textwidth}
\centering
\includegraphics[width=0.9\textwidth]{../charts/08_clustering_esg/chart.pdf}
\end{column}
\end{columns}

\bottomnote{Clustering groups similar ideas, reducing 5000 to meaningful categories}
\end{frame}

% Slide 16: Classification
\begin{frame}{Stage 7: \stagecolor{refine}{Patterns} -- Classification}
\begin{columns}[T]
\begin{column}{0.55\textwidth}
\textbf{Ranking and Categorizing (500)}
\begin{itemize}
\item Identify high-potential patterns
\item Classify by feasibility/impact
\item Rank by multiple criteria
\end{itemize}

\textbf{Signature Equation -- Gini Impurity:}

\sigeq{\text{Gini}(D) = 1 - \sum_{k=1}^{K}p_k^2}

\vspace{0.5em}
\textbf{ESG Application:}

Tier 1: Leaders (verified impact), Tier 2: Improvers, Tier 3: Laggards

\pitfall{Over-relying on historical patterns}
\end{column}

\begin{column}{0.43\textwidth}
\centering
\includegraphics[width=0.9\textwidth]{../charts/09_classification_tiers/chart.pdf}
\end{column}
\end{columns}

\bottomnote{Classification assigns categories based on learned patterns from data}
\end{frame}

% Slide 17: Validation & Metrics
\begin{frame}{Stage 8: \stagecolor{refine}{Insights} -- Validation \& Metrics}
\begin{columns}[T]
\begin{column}{0.55\textwidth}
\textbf{Testing Hypotheses (50)}
\begin{itemize}
\item Rigorous hypothesis testing
\item Compare approaches quantitatively
\item Validate with holdout data
\end{itemize}

\textbf{Signature Equation -- Cross-Validation:}

\sigeq{\text{CV} = \frac{1}{k}\sum_{i=1}^{k}\text{Score}(f_{-i}, D_i)}

\vspace{0.5em}
\textbf{ESG Application:}

5-fold CV on portfolio strategies, compare risk-adjusted returns

\pitfall{Overfitting to historical data}
\end{column}

\begin{column}{0.43\textwidth}
\centering
\includegraphics[width=0.9\textwidth]{../charts/10_validation_metrics/chart.pdf}
\end{column}
\end{columns}

\bottomnote{Validation ensures our insights are genuine, not artifacts of noise}
\end{frame}

% Slide 18: A/B Testing
\begin{frame}{Stage 8b: \stagecolor{refine}{Insights} -- A/B Testing}
\begin{columns}[T]
\begin{column}{0.55\textwidth}
\textbf{Statistical Experimentation}
\begin{itemize}
\item Compare strategies rigorously
\item Control for confounding factors
\item Measure statistical significance
\end{itemize}

\textbf{Signature Equation -- t-statistic:}

\sigeq{t = \frac{\bar{x}_A - \bar{x}_B}{\sqrt{\frac{s_A^2}{n_A} + \frac{s_B^2}{n_B}}}}

\vspace{0.5em}
\textbf{ESG Application:}

A/B test: ESG strategy vs. benchmark on returns, Sharpe ratio, ESG scores

\pitfall{Insufficient sample size}
\end{column}

\begin{column}{0.43\textwidth}
\centering
\textbf{A/B Test Results}

\vspace{0.5em}
\begin{tabular}{lcc}
\toprule
\textbf{Metric} & \textbf{ESG} & \textbf{Bench} \\
\midrule
Returns & 12\% & 9\% \\
Sharpe & 1.2 & 0.9 \\
ESG Score & 78 & 65 \\
Volatility & 15\% & 18\% \\
\bottomrule
\end{tabular}

\vspace{0.5em}
\textit{p-value < 0.05 for returns}
\end{column}
\end{columns}

\bottomnote{A/B Testing provides statistical confidence in strategy comparisons}
\end{frame}

% Slide 19: Responsible AI
\begin{frame}{Stage 9: \stagecolor{strategy}{Strategy} -- Responsible AI}
\begin{columns}[T]
\begin{column}{0.55\textwidth}
\textbf{Final Selection (5)}
\begin{itemize}
\item Apply fairness principles
\item Consider bias and ethics
\item Ensure explainability
\end{itemize}

\textbf{Signature Equation -- SHAP Value:}

\sigeq{\phi_j = \sum_{S}\frac{|S|!(n-|S|-1)!}{n!}[v(S \cup j) - v(S)]}

\vspace{0.5em}
\textbf{ESG Application:}

SHAP explains why each company was selected, ensuring transparency

\pitfall{Black-box decisions without transparency}
\end{column}

\begin{column}{0.43\textwidth}
\centering
\textbf{SHAP Feature Importance}

\vspace{0.5em}
\begin{tabular}{lc}
\toprule
\textbf{Feature} & \textbf{Impact} \\
\midrule
Carbon Score & +0.35 \\
Board Diversity & +0.22 \\
Revenue Growth & +0.18 \\
Controversy Score & -0.15 \\
Sector & +0.10 \\
\bottomrule
\end{tabular}
\end{column}
\end{columns}

\bottomnote{Responsible AI ensures final strategies are fair, explainable, and trustworthy}
\end{frame}

% Slide 20: Structured Output
\begin{frame}[fragile]{Supporting: Structured Output}
\begin{columns}[T]
\begin{column}{0.55\textwidth}
\textbf{Reliable AI Responses}
\begin{itemize}
\item JSON schema validation
\item Consistent output format
\item Production-ready reliability
\end{itemize}

\textbf{Key Concept -- Schema Validation:}

\begin{verbatim}
{
  "strategy": "Climate Leaders",
  "confidence": 0.92,
  "companies": ["MSFT", "AAPL"],
  "risk_level": "medium"
}
\end{verbatim}

\textbf{ESG Application:}

Structured portfolio recommendations with validated JSON output

\pitfall{Unstructured outputs break downstream systems}
\end{column}

\begin{column}{0.43\textwidth}
\centering
\includegraphics[width=0.95\textwidth]{../charts/17_structured_output_reliability/chart.pdf}
\end{column}
\end{columns}

\bottomnote{Structured Output ensures AI responses integrate reliably into production systems}
\end{frame}

% Slide 21: Finance Applications
\begin{frame}{Supporting: Finance Applications}
\begin{columns}[T]
\begin{column}{0.55\textwidth}
\textbf{Quantitative Finance ML}
\begin{itemize}
\item Risk modeling and VaR
\item Portfolio optimization
\item Market prediction
\end{itemize}

\textbf{Signature Equation -- Value at Risk:}

\sigeq{\text{VaR}_\alpha = -\inf\{x : P(X \leq x) \geq \alpha\}}

\vspace{0.5em}
\textbf{ESG Application:}

Efficient frontier with ESG constraints, downside risk modeling

\pitfall{Ignoring tail risks in ESG portfolios}
\end{column}

\begin{column}{0.43\textwidth}
\centering
\includegraphics[width=0.95\textwidth]{../charts/18_efficient_frontier/chart.pdf}
\end{column}
\end{columns}

\bottomnote{Finance Applications bring rigorous quantitative methods to ESG portfolio construction}
\end{frame}

% Slide 22: Convergent Summary
\begin{frame}{Convergent Phase Summary: Techniques That Focus}
\begin{center}
\begin{tabular}{lccc}
\toprule
\textbf{Stage} & \textbf{Count} & \textbf{ML Technique} & \textbf{Course Topic} \\
\midrule
\textcolor{filter}{Extraction} & 2,000 & Clustering & Clustering \\
\textcolor{refine}{Patterns} & 500 & Classification & Classification \\
\textcolor{refine}{Insights} & 50 & Optimization & Validation, A/B Testing \\
\textcolor{strategy}{Strategy} & 5 & Decision Support & Responsible AI, Finance \\
\bottomrule
\end{tabular}
\end{center}

\vspace{0.5em}
\textbf{Key Insight:} Each technique serves a specific focusing purpose

\begin{itemize}
\item \textbf{Clustering} groups and reduces
\item \textbf{Classification} ranks and categorizes
\item \textbf{Validation} tests and confirms
\item \textbf{Responsible AI} ensures quality
\end{itemize}

\bottomnote{Convergent thinking requires ML techniques that filter and focus rather than expand}
\end{frame}

% Slide 23: Visualizing Convergence
\begin{frame}{Visualizing Convergence: 5,000 to 5}
\begin{center}
\includegraphics[width=0.7\textwidth]{../charts/03_convergent_process/chart.pdf}
\end{center}
\bottomnote{The convergent phase systematically reduces possibilities to actionable strategies}
\end{frame}

% ============================================================
% PART 5: SYNTHESIS (8 slides)
% ============================================================

% Slide 24: Final 5 Strategies
\begin{frame}{The Final 5 ESG Strategies}
\begin{center}
\includegraphics[width=0.65\textwidth]{../charts/11_final_strategies/chart.pdf}
\end{center}
\bottomnote{From one challenge to five actionable strategies -- ML enables the full journey}
\end{frame}

% Slide 25: Balance Visual
\begin{frame}{The Key: Balance Is Everything}
\begin{center}
\includegraphics[width=0.55\textwidth]{../charts/13_balance_visual/chart.pdf}
\end{center}

\begin{columns}[T]
\begin{column}{0.48\textwidth}
\textcolor{generate}{\textbf{Too Much Divergence:}}
\begin{itemize}
\item Analysis paralysis
\item No actionable outcomes
\end{itemize}
\end{column}

\begin{column}{0.48\textwidth}
\textcolor{strategy}{\textbf{Too Much Convergence:}}
\begin{itemize}
\item Missed opportunities
\item Local maxima trap
\end{itemize}
\end{column}
\end{columns}

\bottomnote{Successful innovation requires knowing when to expand and when to focus}
\end{frame}

% Slide 26: Complete Journey
\begin{frame}{The Complete ESG Journey}
\begin{center}
\includegraphics[width=0.65\textwidth]{../charts/15_journey_complete/chart.pdf}
\end{center}
\bottomnote{From one challenge to five strategies -- ML enables the full innovation journey}
\end{frame}

% Slide 27: Topic Mapping
\begin{frame}{All 14 Topics on the Diamond}
\begin{center}
\includegraphics[width=0.65\textwidth]{../charts/12_topic_mapping/chart.pdf}
\end{center}
\bottomnote{Every course topic has its place in the innovation journey}
\end{frame}

% Slide 28: When to Use
\begin{frame}{When to Use Each Phase}
\begin{columns}[T]
\begin{column}{0.48\textwidth}
\textcolor{generate}{\textbf{Use Divergent When:}}
\begin{itemize}
\item Problem is new or unclear
\item Need fresh perspectives
\item Current solutions inadequate
\item Early in project lifecycle
\end{itemize}

\vspace{0.5em}
\textbf{ML Tools:}
\begin{itemize}
\item Unsupervised Learning
\item Generative AI
\item NLP \& Topic Modeling
\end{itemize}
\end{column}

\begin{column}{0.48\textwidth}
\textcolor{strategy}{\textbf{Use Convergent When:}}
\begin{itemize}
\item Many options available
\item Resources are limited
\item Decision deadline approaching
\item Ready for implementation
\end{itemize}

\vspace{0.5em}
\textbf{ML Tools:}
\begin{itemize}
\item Clustering \& Classification
\item Validation \& A/B Testing
\item Responsible AI
\end{itemize}
\end{column}
\end{columns}

\bottomnote{Recognize which phase you're in and apply the appropriate ML techniques}
\end{frame}

% Slide 29: Key Message
\begin{frame}{Key Message: Balance Expansion and Focus}
\begin{center}
\Large
\textbf{Innovation = Divergent + Convergent}

\vspace{1.5em}

\begin{tabular}{cc}
\textcolor{generate}{\textbf{DIVERGENT}} & \textcolor{strategy}{\textbf{CONVERGENT}} \\
\hline
Explore possibilities & Filter options \\
Generate ideas & Validate hypotheses \\
Expand the space & Focus on best \\
Creative thinking & Critical thinking \\
\end{tabular}

\vspace{1.5em}

\textit{``The best innovations come from exploring widely, then selecting wisely.''}
\end{center}

\bottomnote{Machine Learning amplifies both sides -- expanding human creativity and sharpening human judgment}
\end{frame}

% Slide 30: ML Toolkit Summary
\begin{frame}{Complete ML Toolkit Summary}
\begin{center}
\small
\begin{tabular}{llll}
\toprule
\textbf{Topic} & \textbf{Phase} & \textbf{Key Equation} & \textbf{Purpose} \\
\midrule
ML Foundations & Divergent & Loss Function & Problem framing \\
Supervised & Divergent & Linear $\hat{y} = X\beta$ & Feature engineering \\
Unsupervised & Divergent & K-Means objective & Pattern discovery \\
Neural Networks & Both & Forward prop & Complex patterns \\
Generative AI & Divergent & $P(x) = \int P(x|z)P(z)dz$ & Idea generation \\
NLP \& Sentiment & Divergent & Language model & Text processing \\
Topic Modeling & Divergent & LDA $P(w|d)$ & Theme extraction \\
Clustering & Convergent & Silhouette score & Grouping \\
Classification & Convergent & Gini impurity & Categorization \\
Validation & Convergent & Cross-validation & Testing \\
A/B Testing & Convergent & t-statistic & Comparison \\
Responsible AI & Convergent & SHAP values & Explainability \\
Structured Output & Convergent & JSON schema & Reliability \\
Finance & Convergent & VaR & Risk modeling \\
\bottomrule
\end{tabular}
\end{center}

\bottomnote{14 tools for the complete innovation journey from challenge to strategy}
\end{frame}

% ============================================================
% PART 6: META-LEARNING & PITFALLS (6 slides)
% ============================================================

% Slide 31: Pitfalls Chart
\begin{frame}{Common Pitfalls by Stage}
\begin{center}
\includegraphics[width=0.7\textwidth]{../charts/14_pitfalls_summary/chart.pdf}
\end{center}
\bottomnote{Awareness of pitfalls at each stage helps navigate the innovation journey successfully}
\end{frame}

% Slide 32: Divergent Pitfalls
\begin{frame}{Pitfalls: Divergent Phase}
\begin{columns}[T]
\begin{column}{0.48\textwidth}
\textbf{\textcolor{challenge}{Challenge Stage}}
\begin{itemize}
\item Too broad: ``Solve climate change''
\item Too narrow: ``Improve this one metric''
\item \textbf{Fix:} Define measurable success
\end{itemize}

\vspace{0.5em}
\textbf{\textcolor{explore}{Exploration Stage}}
\begin{itemize}
\item Ignoring non-obvious dimensions
\item Confirmation bias in data selection
\item \textbf{Fix:} Use unsupervised methods
\end{itemize}
\end{column}

\begin{column}{0.48\textwidth}
\textbf{\textcolor{generate}{Generation Stage}}
\begin{itemize}
\item Quantity without quality filters
\item Hallucinated or infeasible ideas
\item \textbf{Fix:} Structured prompting
\end{itemize}

\vspace{0.5em}
\textbf{\textcolor{peak}{Peak Stage}}
\begin{itemize}
\item Analysis paralysis
\item Lost in the abundance
\item \textbf{Fix:} Set convergence deadline
\end{itemize}
\end{column}
\end{columns}

\bottomnote{Divergent pitfalls often involve losing focus or generating noise instead of signal}
\end{frame}

% Slide 33: Convergent Pitfalls
\begin{frame}{Pitfalls: Convergent Phase}
\begin{columns}[T]
\begin{column}{0.48\textwidth}
\textbf{\textcolor{filter}{Extraction Stage}}
\begin{itemize}
\item Forcing non-existent clusters
\item Wrong number of clusters (k)
\item \textbf{Fix:} Use elbow/silhouette methods
\end{itemize}

\vspace{0.5em}
\textbf{\textcolor{refine}{Pattern Stage}}
\begin{itemize}
\item Over-relying on historical data
\item Overfitting to past patterns
\item \textbf{Fix:} Out-of-sample validation
\end{itemize}
\end{column}

\begin{column}{0.48\textwidth}
\textbf{\textcolor{refine}{Insights Stage}}
\begin{itemize}
\item Lookahead bias in backtesting
\item p-hacking and data snooping
\item \textbf{Fix:} Proper train/test splits
\end{itemize}

\vspace{0.5em}
\textbf{\textcolor{strategy}{Strategy Stage}}
\begin{itemize}
\item Black-box decisions
\item Ignoring ethical implications
\item \textbf{Fix:} SHAP + fairness checks
\end{itemize}
\end{column}
\end{columns}

\bottomnote{Convergent pitfalls often involve premature closure or false confidence}
\end{frame}

% Slide 34: What We Learned
\begin{frame}{What We Learned: Key Insights}
\begin{enumerate}
\item \textbf{ML amplifies human innovation} -- it doesn't replace creativity

\item \textbf{Both phases are essential} -- expansion without focus is chaos; focus without expansion is local maxima

\item \textbf{Match technique to phase} -- use generative tools for divergence, analytical tools for convergence

\item \textbf{Watch for pitfalls} -- each stage has characteristic failure modes

\item \textbf{Trust but verify} -- use validation to confirm ML insights
\end{enumerate}

\vspace{1em}
\begin{center}
\textcolor{mlpurple}{\Large \textbf{1 Challenge $\rightarrow$ 5,000 Ideas $\rightarrow$ 5 Strategies}}
\end{center}

\bottomnote{The Innovation Diamond provides a framework for ML-powered innovation}
\end{frame}

% Slide 35: Decision Framework
\begin{frame}{How to Choose: Decision Framework}
\begin{center}
\textbf{Which ML Technique Should I Use?}
\end{center}

\vspace{0.5em}

\begin{columns}[T]
\begin{column}{0.48\textwidth}
\textbf{If you need to...}
\begin{itemize}
\item Explore unknown structure $\rightarrow$ \textcolor{explore}{Unsupervised}
\item Predict outcomes $\rightarrow$ \textcolor{generate}{Supervised}
\item Generate new content $\rightarrow$ \textcolor{generate}{Generative AI}
\item Process text $\rightarrow$ \textcolor{peak}{NLP}
\item Group similar items $\rightarrow$ \textcolor{filter}{Clustering}
\item Categorize items $\rightarrow$ \textcolor{refine}{Classification}
\item Test hypotheses $\rightarrow$ \textcolor{refine}{A/B Testing}
\end{itemize}
\end{column}

\begin{column}{0.48\textwidth}
\textbf{Key Questions:}
\begin{enumerate}
\item Do you have labels? (Yes $\rightarrow$ Supervised)
\item Are you expanding or focusing?
\item What's your success metric?
\item How much data do you have?
\item Do you need explainability?
\end{enumerate}

\vspace{0.5em}
\textbf{Remember:} No single technique solves everything -- combine approaches!
\end{column}
\end{columns}

\bottomnote{The right ML technique depends on your phase, data, and objectives}
\end{frame}

% Slide 36: Reflection
\begin{frame}{Reflection: Your Innovation Journey}
\begin{center}
\textbf{Think About Your Own Projects:}
\end{center}

\vspace{0.5em}

\begin{enumerate}
\item \textbf{What challenge} are you trying to solve?

\item \textbf{Which phase} are you currently in -- divergent or convergent?

\item \textbf{Which ML techniques} could help you at this stage?

\item \textbf{What pitfalls} should you watch for?

\item \textbf{How will you know} when it's time to switch phases?
\end{enumerate}

\vspace{1em}
\begin{center}
\textit{``The Innovation Diamond is not just a framework -- it's a way of thinking about how ML can augment human creativity and judgment.''}
\end{center}

\bottomnote{Apply these principles to your own innovation challenges}
\end{frame}

% ============================================================
% PART 7: CLOSING (4 slides)
% ============================================================

% Slide 37: Final Takeaways
\begin{frame}{Final Takeaways}
\begin{enumerate}
\item \textbf{ML amplifies human innovation} -- it doesn't replace creativity

\item \textbf{Both phases are essential} -- expansion without focus is chaos; focus without expansion is local maxima

\item \textbf{Match technique to phase} -- use generative tools for divergence, analytical tools for convergence

\item \textbf{Watch for pitfalls} -- each stage has characteristic failure modes

\item \textbf{Trust but verify} -- use validation to confirm ML insights
\end{enumerate}

\vspace{1em}
\begin{center}
\textcolor{mlpurple}{\Large \textbf{1 Challenge $\rightarrow$ 5,000 Ideas $\rightarrow$ 5 Strategies}}

\vspace{0.5em}
\textit{Machine Learning enables both creative expansion and strategic focus}
\end{center}

\bottomnote{The Innovation Diamond provides a framework for ML-powered innovation from challenge to strategy}
\end{frame}

% Slide 38: ESG Summary
\begin{frame}{ESG Case Study: Complete Summary}
\begin{columns}[T]
\begin{column}{0.48\textwidth}
\textbf{The Journey:}
\begin{itemize}
\item \textbf{1} ESG portfolio challenge
\item \textbf{10} sustainability dimensions
\item \textbf{100} engineered features
\item \textbf{1,000} LLM-generated theses
\item \textbf{5,000} raw investment criteria
\item \textbf{2,000} clustered approaches
\item \textbf{500} classified patterns
\item \textbf{50} validated insights
\item \textbf{5} final portfolio strategies
\end{itemize}
\end{column}

\begin{column}{0.48\textwidth}
\textbf{The Strategies:}
\begin{enumerate}
\item Climate Leaders Fund
\item Social Impact Blend
\item Governance Excellence
\item ESG Momentum Strategy
\item Sustainable Dividend Growth
\end{enumerate}

\vspace{0.5em}
\textbf{ML Techniques Used:}

All 14 course topics applied in sequence through the Innovation Diamond
\end{column}
\end{columns}

\bottomnote{A complete demonstration of ML-powered innovation in sustainable finance}
\end{frame}

% Slide 39: Resources
\begin{frame}{Resources \& Next Steps}
\begin{columns}[T]
\begin{column}{0.48\textwidth}
\textbf{Course Materials:}
\begin{itemize}
\item 14 topic slide decks
\item Jupyter notebooks
\item Handouts (basic/intermediate/advanced)
\item Dataset for practice
\end{itemize}

\vspace{0.5em}
\textbf{Key Libraries:}
\begin{itemize}
\item scikit-learn (ML algorithms)
\item transformers (NLP/LLMs)
\item matplotlib/seaborn (visualization)
\end{itemize}
\end{column}

\begin{column}{0.48\textwidth}
\textbf{Practice Projects:}
\begin{enumerate}
\item Apply the Diamond to your own challenge
\item Build an ESG analysis pipeline
\item Create a clustering-based recommender
\item Develop an A/B testing framework
\end{enumerate}

\vspace{0.5em}
\textbf{Remember:}

The best way to learn ML is to \textit{apply it to real problems}!
\end{column}
\end{columns}

\bottomnote{Continue your journey -- apply ML to innovation challenges in your own domain}
\end{frame}

% Slide 40: Thank You
\begin{frame}[plain]
\begin{center}
\vspace{2em}
{\Huge \textcolor{mlpurple}{\textbf{Thank You!}}}

\vspace{2em}
{\Large Machine Learning for Smarter Innovation}

\vspace{1em}
{\large BSc Course Capstone}

\vspace{2em}
\textbf{The Innovation Diamond:}

\vspace{0.5em}
\textcolor{challenge}{1 Challenge} $\rightarrow$
\textcolor{explore}{10} $\rightarrow$
\textcolor{generate}{100} $\rightarrow$
\textcolor{generate}{1,000} $\rightarrow$
\textcolor{peak}{5,000} $\rightarrow$
\textcolor{filter}{2,000} $\rightarrow$
\textcolor{refine}{500} $\rightarrow$
\textcolor{refine}{50} $\rightarrow$
\textcolor{strategy}{5 Strategies}

\vspace{2em}
{\large Questions?}
\end{center}
\end{frame}

\end{document}
