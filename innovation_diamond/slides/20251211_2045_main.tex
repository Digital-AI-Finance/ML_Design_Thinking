\documentclass[8pt,aspectratio=169]{beamer}
\usetheme{Madrid}

% ============================================================
% PACKAGES
% ============================================================
\usepackage{graphicx}
\usepackage{booktabs}
\usepackage{adjustbox}
\usepackage{multicol}
\usepackage{amsmath}
\usepackage{tikz}
\usetikzlibrary{shapes,arrows,positioning}

% ============================================================
% DIAMOND COLOR SCHEME
% ============================================================
% Primary diamond stage colors
\definecolor{challenge}{RGB}{148,103,189}   % Purple - Innovation Challenge
\definecolor{explore}{RGB}{52,152,219}      % Blue - Context Exploration
\definecolor{generate}{RGB}{46,204,113}     % Green - Idea Generation
\definecolor{peak}{RGB}{241,196,15}         % Yellow - Raw Ideas Pool (5000)
\definecolor{filter}{RGB}{230,126,34}       % Orange - Feature Extraction
\definecolor{refine}{RGB}{231,76,60}        % Red - Pattern Discovery
\definecolor{strategy}{RGB}{192,57,43}      % Dark Red - Innovation Strategy

% Supporting colors
\definecolor{mlpurple}{RGB}{51,51,178}      % Primary purple
\definecolor{mllavender}{RGB}{173,173,224}  % Base lavender
\definecolor{mllavender2}{RGB}{193,193,232} % Lighter variant
\definecolor{mllavender3}{RGB}{204,204,235} % Frame title bg
\definecolor{darkgray}{RGB}{64,64,64}       % Text

% ============================================================
% THEME CONFIGURATION
% ============================================================
\setbeamercolor{structure}{fg=mlpurple}
\setbeamercolor{frametitle}{fg=mlpurple,bg=mllavender3}
\setbeamercolor{block title}{fg=mlpurple,bg=mllavender2}
\setbeamercolor{block body}{bg=mllavender3!30}
\setbeamertemplate{navigation symbols}{}
\setbeamertemplate{itemize item}{\textbullet}
\setbeamertemplate{itemize subitem}{--}

% Margins
\setbeamersize{text margin left=5mm,text margin right=5mm}

% ============================================================
% CUSTOM COMMANDS
% ============================================================
\newcommand{\bottomnote}[1]{%
    \vfill
    \textcolor{mllavender2}{\rule{\textwidth}{0.4pt}}
    \par\vspace{2pt}
    {\footnotesize\textcolor{mllavender2}{#1}}
}

\newcommand{\stagecolor}[2]{%
    \textcolor{#1}{\textbf{#2}}
}

\newcommand{\pitfall}[1]{%
    \vspace{0.3em}
    \textcolor{refine}{\textbf{Pitfall:}} \textit{#1}
}

% ============================================================
% DOCUMENT INFO
% ============================================================
\title{ML-Powered Innovation}
\subtitle{From Challenge to Strategy: The Innovation Diamond}
\author{Machine Learning for Smarter Innovation}
\date{BSc Course Capstone}

\begin{document}

% ============================================================
% OPENING (3 slides)
% ============================================================

% Slide 1: Title
\begin{frame}[plain]
\titlepage
\end{frame}

% Slide 2: Course Roadmap
\begin{frame}{Course Roadmap: 14 Topics, One Framework}

\begin{columns}[T]
\begin{column}{0.48\textwidth}
\textbf{Foundations}
\begin{itemize}
\item ML Foundations
\item Supervised Learning
\item Unsupervised Learning
\item Neural Networks
\end{itemize}

\textbf{Core Techniques}
\begin{itemize}
\item Clustering
\item Classification
\item NLP \& Sentiment
\item Topic Modeling
\end{itemize}
\end{column}

\begin{column}{0.48\textwidth}
\textbf{Advanced Applications}
\begin{itemize}
\item Generative AI
\item Structured Output
\item Validation \& Metrics
\item A/B Testing
\end{itemize}

\textbf{Specialized}
\begin{itemize}
\item Responsible AI
\item Finance Applications
\end{itemize}
\end{column}
\end{columns}

\vspace{1em}
\centering
\textcolor{mlpurple}{\textbf{All 14 topics connect through the Innovation Diamond}}

\bottomnote{Each ML technique serves a specific purpose in the innovation journey from challenge to strategy}
\end{frame}

% Slide 3: The Innovation Diamond
\begin{frame}{The Innovation Diamond: From Challenge to Strategy}

\begin{center}
\includegraphics[width=0.75\textwidth]{../charts/01_diamond_overview/chart.pdf}
\end{center}

\bottomnote{Machine Learning enables both creative expansion and strategic focus in innovation}
\end{frame}

% ============================================================
% DIVERGENT PHASE (8 slides)
% ============================================================

% Slide 4: Challenge Stage
\begin{frame}{Stage 1: \stagecolor{challenge}{Innovation Challenge} (1)}

\begin{columns}[T]
\begin{column}{0.55\textwidth}
\textbf{The Starting Point}
\begin{itemize}
\item Define the problem clearly
\item Understand stakeholder needs
\item Set success criteria
\end{itemize}

\textbf{ESG Example:}

\textit{``How can we create an investment portfolio that maximizes returns while ensuring genuine environmental and social impact?''}

\textbf{ML Foundation:} Problem framing determines which algorithms apply

\pitfall{Starting too broad or too narrow -- balance is key}
\end{column}

\begin{column}{0.43\textwidth}
\centering
\includegraphics[width=0.9\textwidth]{../../static/images/topics/ml-foundations/learning_paradigms.png}
\end{column}
\end{columns}

\bottomnote{ML Foundations provides the vocabulary and framework for innovation challenges}
\end{frame}

% Slide 5: Context Exploration
\begin{frame}{Stage 2: \stagecolor{explore}{Context Exploration} (10 dimensions)}

\begin{columns}[T]
\begin{column}{0.55\textwidth}
\textbf{Understanding the Space}
\begin{itemize}
\item Identify relevant dimensions
\item Gather diverse data sources
\item Apply unsupervised exploration
\end{itemize}

\textbf{ESG Example:}

10 dimensions: Carbon emissions, water usage, labor practices, board diversity, supply chain ethics, community impact, waste management, energy efficiency, data privacy, anti-corruption

\textbf{ML Technique:} Data mining, dimensionality analysis

\pitfall{Ignoring non-obvious dimensions that matter}
\end{column}

\begin{column}{0.43\textwidth}
\centering
\includegraphics[width=0.9\textwidth]{../charts/04_esg_dimensions/chart.pdf}
\end{column}
\end{columns}

\bottomnote{Unsupervised Learning reveals hidden structure in complex problem spaces}
\end{frame}

% Slide 6: Feature Discovery
\begin{frame}{Stage 3: \stagecolor{explore}{Feature Discovery} (100 features)}

\begin{columns}[T]
\begin{column}{0.55\textwidth}
\textbf{Extracting Measurable Signals}
\begin{itemize}
\item Transform raw data into features
\item Engineer domain-specific metrics
\item Validate feature relevance
\end{itemize}

\textbf{ESG Example:}

100 features extracted from:
\begin{itemize}
\item Company sustainability reports
\item News sentiment scores
\item Social media mentions
\item Regulatory filings
\end{itemize}

\textbf{ML Technique:} Feature engineering, embeddings

\pitfall{Creating features without domain understanding}
\end{column}

\begin{column}{0.43\textwidth}
\centering
\includegraphics[width=0.9\textwidth]{../charts/05_feature_extraction/chart.pdf}
\end{column}
\end{columns}

\bottomnote{Supervised Learning teaches us which features actually predict outcomes}
\end{frame}

% Slide 7: Idea Generation
\begin{frame}{Stage 4: \stagecolor{generate}{Idea Generation} (1,000 ideas)}

\begin{columns}[T]
\begin{column}{0.55\textwidth}
\textbf{Creative Expansion}
\begin{itemize}
\item Generate diverse possibilities
\item Explore unconventional combinations
\item Push beyond obvious solutions
\end{itemize}

\textbf{ESG Example:}

LLMs generate 1,000 investment thesis variations:
\begin{itemize}
\item ``Invest in companies leading circular economy''
\item ``Target firms with rising ESG momentum''
\item ``Focus on undervalued sustainability leaders''
\end{itemize}

\textbf{ML Technique:} Generative AI, topic modeling

\pitfall{Generating ideas without quality filters}
\end{column}

\begin{column}{0.43\textwidth}
\centering
\includegraphics[width=0.9\textwidth]{../charts/06_idea_generation/chart.pdf}
\end{column}
\end{columns}

\bottomnote{Generative AI and Topic Modeling expand the solution space beyond human capacity}
\end{frame}

% Slide 8: Peak - 5000 Ideas
\begin{frame}{Stage 5: \stagecolor{peak}{Raw Ideas Pool} (5,000 ideas)}

\begin{columns}[T]
\begin{column}{0.55\textwidth}
\textbf{Maximum Expansion}
\begin{itemize}
\item Combine all generated content
\item Include variations and refinements
\item Capture the full possibility space
\end{itemize}

\textbf{ESG Example:}

5,000 potential investment criteria combining:
\begin{itemize}
\item Company targets (500+ firms)
\item Weighting schemes (100+ variations)
\item Sector allocations (50+ strategies)
\item Risk parameters (20+ configurations)
\end{itemize}

\textbf{ML Technique:} NLP for processing, sentiment analysis

\pitfall{Getting lost in the abundance -- analysis paralysis}
\end{column}

\begin{column}{0.43\textwidth}
\centering
\includegraphics[width=0.9\textwidth]{../charts/07_peak_pool/chart.pdf}
\end{column}
\end{columns}

\bottomnote{NLP and Sentiment Analysis help process and understand massive text-based idea pools}
\end{frame}

% Slide 9: Divergent Summary
\begin{frame}{Divergent Phase Summary: Techniques That Expand}

\begin{center}
\begin{tabular}{lccc}
\toprule
\textbf{Stage} & \textbf{Count} & \textbf{ML Technique} & \textbf{Course Topic} \\
\midrule
\textcolor{challenge}{Challenge} & 1 & Problem Framing & ML Foundations \\
\textcolor{explore}{Exploration} & 10 & Data Mining & Unsupervised Learning \\
\textcolor{explore}{Discovery} & 100 & Feature Engineering & Supervised Learning \\
\textcolor{generate}{Generation} & 1,000 & Generative Algorithms & Generative AI \\
\textcolor{peak}{Peak} & 5,000 & NLP Analysis & NLP \& Sentiment \\
\bottomrule
\end{tabular}
\end{center}

\vspace{1em}
\textbf{Key Insight:} Each technique serves a specific expansion purpose

\begin{itemize}
\item \textbf{Unsupervised} learning finds structure without labels
\item \textbf{Generative AI} creates new possibilities
\item \textbf{NLP} processes human-generated content at scale
\end{itemize}

\bottomnote{Divergent thinking requires ML techniques that expand rather than constrain}
\end{frame}

% Slide 10: Divergent Chart
\begin{frame}{Visualizing Expansion: 1 to 5,000}

\begin{center}
\includegraphics[width=0.7\textwidth]{../charts/02_divergent_process/chart.pdf}
\end{center}

\bottomnote{The divergent phase systematically expands from a single challenge to thousands of possibilities}
\end{frame}

% Slide 11: Transition
\begin{frame}{Transition: Now We Must Focus}

\begin{center}
\Large
\textcolor{peak}{\textbf{5,000 ideas}}

\vspace{1em}
$\downarrow$

\vspace{1em}
\textit{``Having many options is valuable only if you can choose wisely.''}

\vspace{1em}
$\downarrow$

\vspace{1em}
\textcolor{strategy}{\textbf{5 strategies}}
\end{center}

\vspace{1em}
The convergent phase applies ML to systematically filter, pattern-match, validate, and select.

\bottomnote{Innovation requires both expansion and focus -- now we apply convergent ML techniques}
\end{frame}

% ============================================================
% CONVERGENT PHASE (8 slides)
% ============================================================

% Slide 12: Feature Extraction
\begin{frame}{Stage 6: \stagecolor{filter}{Feature Extraction} (2,000 filtered)}

\begin{columns}[T]
\begin{column}{0.55\textwidth}
\textbf{Initial Filtering}
\begin{itemize}
\item Group similar ideas
\item Remove duplicates and noise
\item Identify natural clusters
\end{itemize}

\textbf{ESG Example:}

Clustering reveals 8 distinct ESG investment approaches:
\begin{itemize}
\item Best-in-class selectors
\item Exclusionary screens
\item Impact-first strategies
\item ESG momentum plays
\end{itemize}

\textbf{ML Technique:} K-means, DBSCAN, hierarchical clustering

\pitfall{Forcing clusters that don't naturally exist}
\end{column}

\begin{column}{0.43\textwidth}
\centering
\includegraphics[width=0.9\textwidth]{../charts/08_clustering_esg/chart.pdf}
\end{column}
\end{columns}

\bottomnote{Clustering groups similar ideas, reducing 5000 to meaningful categories}
\end{frame}

% Slide 13: Pattern Discovery
\begin{frame}{Stage 7: \stagecolor{refine}{Pattern Discovery} (500 patterns)}

\begin{columns}[T]
\begin{column}{0.55\textwidth}
\textbf{Classification and Ranking}
\begin{itemize}
\item Identify high-potential patterns
\item Classify by feasibility and impact
\item Rank by multiple criteria
\end{itemize}

\textbf{ESG Example:}

Classification into sustainability tiers:
\begin{itemize}
\item Tier 1: Leaders (verified impact)
\item Tier 2: Improvers (positive trajectory)
\item Tier 3: Laggards (material risks)
\end{itemize}

\textbf{ML Technique:} Decision trees, random forests

\pitfall{Over-relying on historical patterns for future decisions}
\end{column}

\begin{column}{0.43\textwidth}
\centering
\includegraphics[width=0.9\textwidth]{../charts/09_classification_tiers/chart.pdf}
\end{column}
\end{columns}

\bottomnote{Classification assigns categories based on learned patterns from data}
\end{frame}

% Slide 14: Refined Insights
\begin{frame}{Stage 8: \stagecolor{refine}{Refined Insights} (50 insights)}

\begin{columns}[T]
\begin{column}{0.55\textwidth}
\textbf{Validation and Testing}
\begin{itemize}
\item Test hypotheses rigorously
\item Compare approaches quantitatively
\item Validate with holdout data
\end{itemize}

\textbf{ESG Example:}

A/B testing portfolio strategies:
\begin{itemize}
\item Backtest on historical data
\item Compare risk-adjusted returns
\item Measure ESG score improvements
\end{itemize}

\textbf{ML Technique:} Cross-validation, statistical testing

\pitfall{Overfitting to historical data (lookahead bias)}
\end{column}

\begin{column}{0.43\textwidth}
\centering
\includegraphics[width=0.9\textwidth]{../charts/10_validation_metrics/chart.pdf}
\end{column}
\end{columns}

\bottomnote{Validation and A/B Testing ensure our insights are genuine, not artifacts of noise}
\end{frame}

% Slide 15: Innovation Strategy
\begin{frame}{Stage 9: \stagecolor{strategy}{Innovation Strategy} (5 strategies)}

\begin{columns}[T]
\begin{column}{0.55\textwidth}
\textbf{Final Selection}
\begin{itemize}
\item Apply responsible AI principles
\item Consider fairness and bias
\item Ensure explainability
\end{itemize}

\textbf{ESG Example:}

5 final portfolio strategies:
\begin{enumerate}
\item Climate Leaders Fund
\item Social Impact Blend
\item Governance Excellence
\item ESG Momentum Strategy
\item Sustainable Dividend Growth
\end{enumerate}

\textbf{ML Technique:} SHAP explanations, fairness metrics

\pitfall{Black-box decisions without transparency}
\end{column}

\begin{column}{0.43\textwidth}
\centering
\includegraphics[width=0.9\textwidth]{../charts/11_final_strategies/chart.pdf}
\end{column}
\end{columns}

\bottomnote{Responsible AI ensures final strategies are fair, explainable, and trustworthy}
\end{frame}

% Slide 16: Convergent Summary
\begin{frame}{Convergent Phase Summary: Techniques That Focus}

\begin{center}
\begin{tabular}{lccc}
\toprule
\textbf{Stage} & \textbf{Count} & \textbf{ML Technique} & \textbf{Course Topic} \\
\midrule
\textcolor{filter}{Extraction} & 2,000 & Clustering & Clustering \\
\textcolor{refine}{Patterns} & 500 & Classification & Classification \\
\textcolor{refine}{Insights} & 50 & Optimization & Validation \& A/B \\
\textcolor{strategy}{Strategy} & 5 & Decision Support & Responsible AI \\
\bottomrule
\end{tabular}
\end{center}

\vspace{1em}
\textbf{Key Insight:} Each technique serves a specific focusing purpose

\begin{itemize}
\item \textbf{Clustering} groups and reduces
\item \textbf{Classification} ranks and categorizes
\item \textbf{Validation} tests and confirms
\item \textbf{Responsible AI} ensures quality
\end{itemize}

\bottomnote{Convergent thinking requires ML techniques that filter and focus rather than expand}
\end{frame}

% Slide 17: Convergent Chart
\begin{frame}{Visualizing Convergence: 5,000 to 5}

\begin{center}
\includegraphics[width=0.7\textwidth]{../charts/03_convergent_process/chart.pdf}
\end{center}

\bottomnote{The convergent phase systematically reduces possibilities to actionable strategies}
\end{frame}

% Slide 18: Balance Transition
\begin{frame}{The Key: Balance Is Everything}

\begin{center}
\includegraphics[width=0.65\textwidth]{../charts/13_balance_visual/chart.pdf}
\end{center}

\vspace{0.5em}
\begin{columns}[T]
\begin{column}{0.48\textwidth}
\textbf{Too Much Divergence:}
\begin{itemize}
\item Analysis paralysis
\item No actionable outcomes
\item Wasted resources
\end{itemize}
\end{column}

\begin{column}{0.48\textwidth}
\textbf{Too Much Convergence:}
\begin{itemize}
\item Premature optimization
\item Missed opportunities
\item Local maxima trap
\end{itemize}
\end{column}
\end{columns}

\bottomnote{Successful innovation requires knowing when to expand and when to focus}
\end{frame}

% ============================================================
% SYNTHESIS (4 slides)
% ============================================================

% Slide 19: Full Journey
\begin{frame}{The Complete ESG Journey}

\begin{center}
\includegraphics[width=0.8\textwidth]{../charts/15_journey_complete/chart.pdf}
\end{center}

\bottomnote{From one challenge to five strategies -- ML enables the full innovation journey}
\end{frame}

% Slide 20: Key Message
\begin{frame}{Key Message: Balance Expansion and Focus}

\begin{center}
\Large
\textbf{Innovation = Divergent + Convergent}

\vspace{2em}

\begin{tabular}{cc}
\textcolor{generate}{\textbf{DIVERGENT}} & \textcolor{strategy}{\textbf{CONVERGENT}} \\
\hline
Explore possibilities & Filter options \\
Generate ideas & Validate hypotheses \\
Expand the space & Focus on best \\
Creative thinking & Critical thinking \\
\end{tabular}

\vspace{2em}

\textit{``The best innovations come from exploring widely, then selecting wisely.''}
\end{center}

\bottomnote{Machine Learning amplifies both sides -- expanding human creativity and sharpening human judgment}
\end{frame}

% Slide 21: Course Summary
\begin{frame}{All 14 Topics on the Diamond}

\begin{center}
\includegraphics[width=0.75\textwidth]{../charts/12_topic_mapping/chart.pdf}
\end{center}

\bottomnote{Every course topic has its place in the innovation journey}
\end{frame}

% Slide 22: When to Use
\begin{frame}{When to Use Each Phase}

\begin{columns}[T]
\begin{column}{0.48\textwidth}
\textcolor{generate}{\textbf{Use Divergent When:}}
\begin{itemize}
\item Problem is new or unclear
\item Need fresh perspectives
\item Current solutions inadequate
\item Exploring new markets
\item Early in project lifecycle
\end{itemize}

\vspace{0.5em}
\textbf{ML Tools:}
\begin{itemize}
\item Unsupervised Learning
\item Generative AI
\item NLP \& Topic Modeling
\end{itemize}
\end{column}

\begin{column}{0.48\textwidth}
\textcolor{strategy}{\textbf{Use Convergent When:}}
\begin{itemize}
\item Many options available
\item Resources are limited
\item Decision deadline approaching
\item Need to prioritize
\item Ready for implementation
\end{itemize}

\vspace{0.5em}
\textbf{ML Tools:}
\begin{itemize}
\item Clustering \& Classification
\item Validation \& A/B Testing
\item Responsible AI
\end{itemize}
\end{column}
\end{columns}

\bottomnote{Recognize which phase you're in and apply the appropriate ML techniques}
\end{frame}

% ============================================================
% CLOSING (2 slides)
% ============================================================

% Slide 23: Pitfalls Summary
\begin{frame}{Common Pitfalls Summary}

\begin{center}
\includegraphics[width=0.7\textwidth]{../charts/14_pitfalls_summary/chart.pdf}
\end{center}

\bottomnote{Awareness of pitfalls at each stage helps navigate the innovation journey successfully}
\end{frame}

% Slide 24: Final Takeaways
\begin{frame}{Final Takeaways}

\begin{enumerate}
\item \textbf{ML amplifies human innovation} -- it doesn't replace creativity
\item \textbf{Both phases are essential} -- expansion without focus is chaos; focus without expansion is local maxima
\item \textbf{Match technique to phase} -- use generative tools for divergence, analytical tools for convergence
\item \textbf{Watch for pitfalls} -- each stage has characteristic failure modes
\item \textbf{Trust but verify} -- use validation to confirm ML insights
\end{enumerate}

\vspace{1em}
\begin{center}
\textcolor{mlpurple}{\Large \textbf{1 Challenge $\rightarrow$ 5,000 Ideas $\rightarrow$ 5 Strategies}}

\vspace{0.5em}
\textit{Machine Learning enables both creative expansion and strategic focus}
\end{center}

\bottomnote{The Innovation Diamond provides a framework for ML-powered innovation from challenge to strategy}
\end{frame}

\end{document}
