% Pandoc LaTeX template for course handouts
% Usage: pandoc input.md -o output.tex --template=handout_template.tex --listings
\documentclass[10pt,a4paper]{article}

% Geometry
\usepackage[margin=2.5cm]{geometry}

% Fonts and encoding
\usepackage[T1]{fontenc}
\usepackage[utf8]{inputenc}
\usepackage{lmodern}

% Math
\usepackage{amsmath,amssymb,amsfonts}

% Tables
\usepackage{booktabs}
\usepackage{longtable}
\usepackage{array}

% Graphics
\usepackage{graphicx}

% Colors
\usepackage{xcolor}
\definecolor{mlpurple}{RGB}{51,51,178}
\definecolor{mlblue}{RGB}{0,102,204}
\definecolor{codebg}{RGB}{248,248,248}
\definecolor{codeframe}{RGB}{200,200,200}

% Code listings
\usepackage{listings}
\lstset{
  backgroundcolor=\color{codebg},
  basicstyle=\ttfamily\small,
  breakatwhitespace=false,
  breaklines=true,
  captionpos=b,
  commentstyle=\color{gray},
  frame=single,
  framerule=0.5pt,
  rulecolor=\color{codeframe},
  keepspaces=true,
  keywordstyle=\color{mlblue}\bfseries,
  language=Python,
  numbers=none,
  showspaces=false,
  showstringspaces=false,
  showtabs=false,
  stringstyle=\color{mlpurple},
  tabsize=4,
  xleftmargin=0.5em,
  xrightmargin=0.5em,
  aboveskip=1em,
  belowskip=1em
}

% Hyperlinks
\usepackage{hyperref}
\hypersetup{
  colorlinks=true,
  linkcolor=mlblue,
  urlcolor=mlblue,
  citecolor=mlblue
}

% Headers and footers
\usepackage{fancyhdr}
\pagestyle{fancy}
\fancyhf{}
\lhead{\textcolor{mlpurple}{\textbf{Unsupervised Learning}}}
\rhead{\textcolor{gray}{Basic}}
\cfoot{\thepage}
\renewcommand{\headrulewidth}{0.4pt}
\renewcommand{\footrulewidth}{0pt}

% Paragraph spacing
\setlength{\parindent}{0pt}
\setlength{\parskip}{0.5em}

% Section formatting
\usepackage{titlesec}
\titleformat{\section}
  {\Large\bfseries\color{mlpurple}}
  {\thesection}{1em}{}
\titleformat{\subsection}
  {\large\bfseries\color{mlblue}}
  {\thesubsection}{1em}{}
\titleformat{\subsubsection}
  {\normalsize\bfseries}
  {\thesubsubsection}{1em}{}

% Lists
\usepackage{enumitem}
\setlist[itemize]{topsep=0.5em, itemsep=0.25em}
\setlist[enumerate]{topsep=0.5em, itemsep=0.25em}

% Tight list support for pandoc
\providecommand{\tightlist}{%
  \setlength{\itemsep}{0pt}\setlength{\parskip}{0pt}}

% Title
\title{\Huge\bfseries\color{mlpurple} Unsupervised Learning - Basic Handout}
\author{Machine Learning for Smarter Innovation}
\date{}

\begin{document}

\maketitle
\thispagestyle{fancy}

\section{Unsupervised Learning - Basic
Handout}\label{unsupervised-learning---basic-handout}

\textbf{Target Audience}: Beginners with no ML background
\textbf{Duration}: 25 minutes reading \textbf{Level}: Basic (no math
required)

\begin{center}\rule{0.5\linewidth}{0.5pt}\end{center}

\subsection{What Is Unsupervised
Learning?}\label{what-is-unsupervised-learning}

Think of it like organizing a messy closet. Nobody tells you which items
go together - you discover patterns yourself based on colors, seasons,
or occasions.

\textbf{Supervised vs Unsupervised}: - \textbf{Supervised}: Teacher
gives you answers to learn from - \textbf{Unsupervised}: No answers -
find patterns on your own

\begin{center}\rule{0.5\linewidth}{0.5pt}\end{center}

\subsection{Real-World Examples}\label{real-world-examples}

\subsubsection{Customer Segmentation}\label{customer-segmentation}

\begin{itemize}
\tightlist
\item
  Group customers by behavior (not demographics)
\item
  Discover: ``Weekend browsers'' vs ``Quick buyers'' vs ``Sale hunters''
\end{itemize}

\subsubsection{Anomaly Detection}\label{anomaly-detection}

\begin{itemize}
\tightlist
\item
  Find unusual transactions (fraud detection)
\item
  Spot equipment failures before they happen
\end{itemize}

\subsubsection{Document Organization}\label{document-organization}

\begin{itemize}
\tightlist
\item
  Group news articles by topic automatically
\item
  Organize customer feedback into themes
\end{itemize}

\subsubsection{Recommendation Systems}\label{recommendation-systems}

\begin{itemize}
\tightlist
\item
  ``Customers like you also bought\ldots{}''
\item
  Spotify's Discover Weekly playlists
\end{itemize}

\begin{center}\rule{0.5\linewidth}{0.5pt}\end{center}

\subsection{Three Main Types}\label{three-main-types}

\subsubsection{1. Clustering}\label{clustering}

\textbf{Goal}: Group similar items together - K-means: You choose number
of groups - DBSCAN: Algorithm finds natural groupings - Hierarchical:
Creates tree of relationships

\textbf{Use when}: You want to find natural segments

\subsubsection{2. Dimensionality
Reduction}\label{dimensionality-reduction}

\textbf{Goal}: Simplify complex data while keeping patterns - PCA: Find
most important features - t-SNE: Create visual maps of high-dimensional
data

\textbf{Use when}: Too many features to analyze

\subsubsection{3. Association Rules}\label{association-rules}

\textbf{Goal}: Find items that appear together - Market basket: ``People
who buy X also buy Y'' - Example: Diapers and beer on Friday evenings

\textbf{Use when}: Finding hidden relationships

\begin{center}\rule{0.5\linewidth}{0.5pt}\end{center}

\subsection{When to Use Unsupervised
Learning}\label{when-to-use-unsupervised-learning}

\subsubsection{Good Fit:}\label{good-fit}

\begin{itemize}
\tightlist
\item
  Exploring new datasets
\item
  No labeled training data available
\item
  Looking for hidden patterns
\item
  Reducing data complexity
\item
  Generating features for other models
\end{itemize}

\subsubsection{Poor Fit:}\label{poor-fit}

\begin{itemize}
\tightlist
\item
  Need specific predictions
\item
  Have clear right/wrong answers
\item
  Very small datasets (under 100 items)
\item
  Need highly interpretable results
\end{itemize}

\begin{center}\rule{0.5\linewidth}{0.5pt}\end{center}

\subsection{Quick Start Checklist}\label{quick-start-checklist}

\subsubsection{Before You Begin:}\label{before-you-begin}

\begin{itemize}
\tightlist
\item[$\square$]
  Define your exploration goal
\item[$\square$]
  Ensure data is cleaned and scaled
\item[$\square$]
  Remove or handle missing values
\item[$\square$]
  Identify which features to include
\end{itemize}

\subsubsection{Your First Project:}\label{your-first-project}

\begin{itemize}
\tightlist
\item[$\square$]
  Start with K-means clustering (K=3)
\item[$\square$]
  Visualize results with scatter plots
\item[$\square$]
  Try different K values (2-10)
\item[$\square$]
  Name your discovered groups
\end{itemize}

\subsubsection{Validate Results:}\label{validate-results}

\begin{itemize}
\tightlist
\item[$\square$]
  Do groups make business sense?
\item[$\square$]
  Are groups distinct and meaningful?
\item[$\square$]
  Can you take action on findings?
\end{itemize}

\begin{center}\rule{0.5\linewidth}{0.5pt}\end{center}

\subsection{Common Pitfalls}\label{common-pitfalls}

\begin{enumerate}
\def\labelenumi{\arabic{enumi}.}
\tightlist
\item
  \textbf{Forgetting to scale data}: Features with larger values
  dominate
\item
  \textbf{Choosing K randomly}: Use elbow method or silhouette score
\item
  \textbf{Ignoring outliers}: They can distort cluster centers
\item
  \textbf{Over-interpreting}: Not all patterns are meaningful
\item
  \textbf{No domain validation}: Always check with experts
\end{enumerate}

\begin{center}\rule{0.5\linewidth}{0.5pt}\end{center}

\subsection{Tools for Beginners}\label{tools-for-beginners}

\subsubsection{No-Code Options:}\label{no-code-options}

\begin{itemize}
\tightlist
\item
  \textbf{Orange3}: Visual drag-and-drop ML
\item
  \textbf{KNIME}: Workflow-based analytics
\item
  \textbf{Tableau}: Built-in clustering
\end{itemize}

\subsubsection{Python Libraries:}\label{python-libraries}

\begin{itemize}
\tightlist
\item
  \textbf{scikit-learn}: Standard ML library
\item
  \textbf{pandas}: Data manipulation
\item
  \textbf{matplotlib/seaborn}: Visualization
\end{itemize}

\begin{center}\rule{0.5\linewidth}{0.5pt}\end{center}

\subsection{Key Terms}\label{key-terms}

\begin{longtable}[]{@{}ll@{}}
\toprule\noalign{}
Term & Simple Definition \\
\midrule\noalign{}
\endhead
\bottomrule\noalign{}
\endlastfoot
Clustering & Grouping similar items \\
K-means & Choose K groups, find centers \\
DBSCAN & Density-based grouping \\
PCA & Reduce dimensions, keep variance \\
Silhouette & Quality score (-1 to 1, higher better) \\
Elbow method & Graph to choose optimal K \\
\end{longtable}

\begin{center}\rule{0.5\linewidth}{0.5pt}\end{center}

\subsection{Next Steps}\label{next-steps}

\begin{enumerate}
\def\labelenumi{\arabic{enumi}.}
\tightlist
\item
  \textbf{Try}: Cluster a simple dataset (iris, wine)
\item
  \textbf{Explore}: Visualize your clusters
\item
  \textbf{Compare}: Try K=2, 3, 4, 5 and compare
\item
  \textbf{Read}: Intermediate handout for implementation details
\end{enumerate}

\begin{center}\rule{0.5\linewidth}{0.5pt}\end{center}

\emph{Unsupervised learning is about discovery. Let the data tell its
story, then use your expertise to interpret what it means.}

\end{document}
