% Pandoc LaTeX template for course handouts
% Usage: pandoc input.md -o output.tex --template=handout_template.tex --listings
\documentclass[10pt,a4paper]{article}

% Geometry
\usepackage[margin=2.5cm]{geometry}

% Fonts and encoding
\usepackage[T1]{fontenc}
\usepackage[utf8]{inputenc}
\usepackage{lmodern}

% Math
\usepackage{amsmath,amssymb,amsfonts}

% Tables
\usepackage{booktabs}
\usepackage{longtable}
\usepackage{array}

% Graphics
\usepackage{graphicx}

% Colors
\usepackage{xcolor}
\definecolor{mlpurple}{RGB}{51,51,178}
\definecolor{mlblue}{RGB}{0,102,204}
\definecolor{codebg}{RGB}{248,248,248}
\definecolor{codeframe}{RGB}{200,200,200}

% Code listings
\usepackage{listings}
\lstset{
  backgroundcolor=\color{codebg},
  basicstyle=\ttfamily\small,
  breakatwhitespace=false,
  breaklines=true,
  captionpos=b,
  commentstyle=\color{gray},
  frame=single,
  framerule=0.5pt,
  rulecolor=\color{codeframe},
  keepspaces=true,
  keywordstyle=\color{mlblue}\bfseries,
  language=Python,
  numbers=none,
  showspaces=false,
  showstringspaces=false,
  showtabs=false,
  stringstyle=\color{mlpurple},
  tabsize=4,
  xleftmargin=0.5em,
  xrightmargin=0.5em,
  aboveskip=1em,
  belowskip=1em
}

% Hyperlinks
\usepackage{hyperref}
\hypersetup{
  colorlinks=true,
  linkcolor=mlblue,
  urlcolor=mlblue,
  citecolor=mlblue
}

% Headers and footers
\usepackage{fancyhdr}
\pagestyle{fancy}
\fancyhf{}
\lhead{\textcolor{mlpurple}{\textbf{Clustering}}}
\rhead{\textcolor{gray}{Basic}}
\cfoot{\thepage}
\renewcommand{\headrulewidth}{0.4pt}
\renewcommand{\footrulewidth}{0pt}

% Paragraph spacing
\setlength{\parindent}{0pt}
\setlength{\parskip}{0.5em}

% Section formatting
\usepackage{titlesec}
\titleformat{\section}
  {\Large\bfseries\color{mlpurple}}
  {\thesection}{1em}{}
\titleformat{\subsection}
  {\large\bfseries\color{mlblue}}
  {\thesubsection}{1em}{}
\titleformat{\subsubsection}
  {\normalsize\bfseries}
  {\thesubsubsection}{1em}{}

% Lists
\usepackage{enumitem}
\setlist[itemize]{topsep=0.5em, itemsep=0.25em}
\setlist[enumerate]{topsep=0.5em, itemsep=0.25em}

% Tight list support for pandoc
\providecommand{\tightlist}{%
  \setlength{\itemsep}{0pt}\setlength{\parskip}{0pt}}

% Title
\title{\Huge\bfseries\color{mlpurple} Week 1 Handout 1: Basic Clustering Fundamentals}
\author{Machine Learning for Smarter Innovation}
\date{}

\begin{document}

\maketitle
\thispagestyle{fancy}

\section{Week 1 Handout 1: Basic Clustering
Fundamentals}\label{week-1-handout-1-basic-clustering-fundamentals}

\textbf{Target Audience}: Beginners with no ML background
\textbf{Duration}: 30 minutes reading \textbf{Level}: Basic

\begin{center}\rule{0.5\linewidth}{0.5pt}\end{center}

\subsection{What Is Clustering?}\label{what-is-clustering}

Think of clustering like organizing your music collection. Instead of
having thousands of songs scattered randomly, you group them by genre,
mood, or artist. Clustering does the same thing with data - it finds
natural groups automatically.

\subsubsection{Real-World Examples}\label{real-world-examples}

\begin{itemize}
\tightlist
\item
  \textbf{Netflix}: Groups movies by viewing patterns (not just genre)
\item
  \textbf{Spotify}: Creates playlists based on listening habits
\item
  \textbf{Amazon}: Groups customers by shopping behavior
\item
  \textbf{Marketing}: Segments customers for targeted campaigns
\end{itemize}

\subsubsection{Why Clustering Matters for
Innovation}\label{why-clustering-matters-for-innovation}

\begin{itemize}
\tightlist
\item
  \textbf{Scale}: Handle thousands of ideas instead of dozens
\item
  \textbf{Objectivity}: Removes human bias from grouping
\item
  \textbf{Discovery}: Finds patterns you never noticed
\item
  \textbf{Speed}: Minutes instead of weeks for analysis
\end{itemize}

\begin{center}\rule{0.5\linewidth}{0.5pt}\end{center}

\subsection{Key Concepts (No Math
Required)}\label{key-concepts-no-math-required}

\subsubsection{1. What Makes Things
Similar?}\label{what-makes-things-similar}

Clustering looks at features (characteristics) to decide what goes
together: - \textbf{Customer example}: Age, income, location, spending
habits - \textbf{Product example}: Price, category, ratings, reviews -
\textbf{Innovation example}: Market size, technology level, funding
needs

\subsubsection{2. How Many Groups?}\label{how-many-groups}

This is the \textbf{K} in K-means (the most popular method): -
\textbf{Too few groups}: Everything mixed together (not useful) -
\textbf{Too many groups}: Tiny groups (overwhelming) - \textbf{Just
right}: Clear, actionable segments

\subsubsection{3. Quality Check}\label{quality-check}

How do you know if your groups are good? - \textbf{Tight groups}: Items
in same group are very similar - \textbf{Separate groups}: Different
groups are clearly distinct - \textbf{Makes sense}: Groups tell a
meaningful story

\begin{center}\rule{0.5\linewidth}{0.5pt}\end{center}

\subsection{The K-Means Process (Simple
Version)}\label{the-k-means-process-simple-version}

\subsubsection{Step 1: Choose Number of Groups
(K)}\label{step-1-choose-number-of-groups-k}

\begin{itemize}
\tightlist
\item
  Start with your best guess
\item
  Common rule: Try 3-5 groups first
\item
  You can adjust later
\end{itemize}

\subsubsection{Step 2: Let the Computer Find
Groups}\label{step-2-let-the-computer-find-groups}

\begin{itemize}
\tightlist
\item
  Algorithm places starting points randomly
\item
  Assigns each item to nearest starting point
\item
  Moves starting points to center of their groups
\item
  Repeats until groups stabilize
\end{itemize}

\subsubsection{Step 3: Check Quality}\label{step-3-check-quality}

\begin{itemize}
\tightlist
\item
  Look at the results visually
\item
  Use quality scores (like grades)
\item
  Ask: ``Do these groups make business sense?''
\end{itemize}

\subsubsection{Step 4: Name Your Groups}\label{step-4-name-your-groups}

\begin{itemize}
\tightlist
\item
  \textbf{Cluster 1}: ``Tech Innovators'' (high funding, software focus)
\item
  \textbf{Cluster 2}: ``Bootstrap Builders'' (low funding, service
  focus)
\item
  \textbf{Cluster 3}: ``Green Pioneers'' (sustainability focus)
\end{itemize}

\begin{center}\rule{0.5\linewidth}{0.5pt}\end{center}

\subsection{When NOT to Use
Clustering}\label{when-not-to-use-clustering}

\subsubsection{Clustering Won't Help
When:}\label{clustering-wont-help-when}

\begin{itemize}
\tightlist
\item
  You already know your exact groups
\item
  You have very little data (under 50 items)
\item
  All your data looks the same
\item
  You need to predict specific outcomes (use classification instead)
\end{itemize}

\subsubsection{Common Mistakes to
Avoid:}\label{common-mistakes-to-avoid}

\begin{itemize}
\tightlist
\item
  Not preparing data properly
\item
  Choosing too many groups
\item
  Ignoring domain knowledge
\item
  Over-interpreting results
\end{itemize}

\begin{center}\rule{0.5\linewidth}{0.5pt}\end{center}

\subsection{Getting Started Checklist}\label{getting-started-checklist}

\subsubsection{Before You Begin:}\label{before-you-begin}

\begin{itemize}
\tightlist
\item[$\square$]
  \textbf{Clear goal}: What question are you trying to answer?
\item[$\square$]
  \textbf{Clean data}: Remove errors, handle missing values
\item[$\square$]
  \textbf{Right features}: Choose characteristics that matter
\item[$\square$]
  \textbf{Enough data}: At least 100 items recommended
\end{itemize}

\subsubsection{For Your First Project:}\label{for-your-first-project}

\begin{itemize}
\tightlist
\item[$\square$]
  \textbf{Start simple}: Use K-means with 3-5 groups
\item[$\square$]
  \textbf{Visualize}: Create charts to see patterns
\item[$\square$]
  \textbf{Validate}: Check if results make sense
\item[$\square$]
  \textbf{Iterate}: Try different numbers of groups
\end{itemize}

\subsubsection{Success Indicators:}\label{success-indicators}

\begin{itemize}
\tightlist
\item[$\square$]
  \textbf{Clear separation}: Groups look distinct
\item[$\square$]
  \textbf{Business relevance}: Groups tell meaningful stories
\item[$\square$]
  \textbf{Actionable insights}: You can make decisions based on groups
\item[$\square$]
  \textbf{Stable results}: Groups don't change dramatically with small
  data changes
\end{itemize}

\begin{center}\rule{0.5\linewidth}{0.5pt}\end{center}

\subsection{Tools for Beginners}\label{tools-for-beginners}

\subsubsection{No-Code Options:}\label{no-code-options}

\begin{itemize}
\tightlist
\item
  \textbf{Excel}: Basic clustering with scatter plots
\item
  \textbf{Google Sheets}: Simple data grouping
\item
  \textbf{Tableau}: Visual clustering analysis
\item
  \textbf{Orange3}: Drag-and-drop ML tool
\end{itemize}

\subsubsection{When You're Ready for
Code:}\label{when-youre-ready-for-code}

\begin{itemize}
\tightlist
\item
  \textbf{Python}: Most popular for clustering
\item
  \textbf{R}: Great for statistical analysis
\item
  \textbf{SPSS}: User-friendly statistical software
\end{itemize}

\begin{center}\rule{0.5\linewidth}{0.5pt}\end{center}

\subsection{Next Steps}\label{next-steps}

\subsubsection{This Week:}\label{this-week}

\begin{enumerate}
\def\labelenumi{\arabic{enumi}.}
\tightlist
\item
  \textbf{Identify} a dataset you want to explore
\item
  \textbf{Think} about what groups might exist
\item
  \textbf{Try} the practice exercise
\item
  \textbf{Join} the Slack discussion
\end{enumerate}

\subsubsection{Next Week Preview:}\label{next-week-preview}

\begin{itemize}
\tightlist
\item
  Advanced clustering techniques
\item
  Handling complex data types
\item
  Real industry applications
\item
  Building automated pipelines
\end{itemize}

\begin{center}\rule{0.5\linewidth}{0.5pt}\end{center}

\subsection{Questions to Ask Yourself}\label{questions-to-ask-yourself}

\begin{enumerate}
\def\labelenumi{\arabic{enumi}.}
\tightlist
\item
  \textbf{Data}: What characteristics define similarity in my domain?
\item
  \textbf{Groups}: How many natural segments do I expect?
\item
  \textbf{Purpose}: What decisions will these groups help me make?
\item
  \textbf{Validation}: How will I know if the results are good?
\item
  \textbf{Action}: What will I do differently based on these groups?
\end{enumerate}

\begin{center}\rule{0.5\linewidth}{0.5pt}\end{center}

\subsection{Quick Reference}\label{quick-reference}

\subsubsection{Key Terms:}\label{key-terms}

\begin{itemize}
\tightlist
\item
  \textbf{Clustering}: Grouping similar items automatically
\item
  \textbf{K-means}: Most popular clustering method
\item
  \textbf{K}: Number of groups you want
\item
  \textbf{Features}: Characteristics used for grouping
\item
  \textbf{Centroid}: Center point of a group
\end{itemize}

\subsubsection{Success Metrics:}\label{success-metrics}

\begin{itemize}
\tightlist
\item
  \textbf{Silhouette Score}: Quality measure (higher = better)
\item
  \textbf{Elbow Method}: Helps choose optimal number of groups
\item
  \textbf{Business Validation}: Do results make practical sense?
\end{itemize}

\begin{center}\rule{0.5\linewidth}{0.5pt}\end{center}

\emph{Remember: Clustering is a tool for discovery, not a magic
solution. The insights come from combining algorithmic results with
human expertise and domain knowledge.}

\end{document}
