% Pandoc LaTeX template for course handouts
% Usage: pandoc input.md -o output.tex --template=handout_template.tex --listings
\documentclass[10pt,a4paper]{article}

% Geometry
\usepackage[margin=2.5cm]{geometry}

% Fonts and encoding
\usepackage[T1]{fontenc}
\usepackage[utf8]{inputenc}
\usepackage{lmodern}

% Math
\usepackage{amsmath,amssymb,amsfonts}

% Tables
\usepackage{booktabs}
\usepackage{longtable}
\usepackage{array}

% Graphics
\usepackage{graphicx}

% Colors
\usepackage{xcolor}
\definecolor{mlpurple}{RGB}{51,51,178}
\definecolor{mlblue}{RGB}{0,102,204}
\definecolor{codebg}{RGB}{248,248,248}
\definecolor{codeframe}{RGB}{200,200,200}

% Code listings
\usepackage{listings}
\lstset{
  backgroundcolor=\color{codebg},
  basicstyle=\ttfamily\small,
  breakatwhitespace=false,
  breaklines=true,
  captionpos=b,
  commentstyle=\color{gray},
  frame=single,
  framerule=0.5pt,
  rulecolor=\color{codeframe},
  keepspaces=true,
  keywordstyle=\color{mlblue}\bfseries,
  language=Python,
  numbers=none,
  showspaces=false,
  showstringspaces=false,
  showtabs=false,
  stringstyle=\color{mlpurple},
  tabsize=4,
  xleftmargin=0.5em,
  xrightmargin=0.5em,
  aboveskip=1em,
  belowskip=1em
}

% Hyperlinks
\usepackage{hyperref}
\hypersetup{
  colorlinks=true,
  linkcolor=mlblue,
  urlcolor=mlblue,
  citecolor=mlblue
}

% Headers and footers
\usepackage{fancyhdr}
\pagestyle{fancy}
\fancyhf{}
\lhead{\textcolor{mlpurple}{\textbf{Finance Applications}}}
\rhead{\textcolor{gray}{Basic}}
\cfoot{\thepage}
\renewcommand{\headrulewidth}{0.4pt}
\renewcommand{\footrulewidth}{0pt}

% Paragraph spacing
\setlength{\parindent}{0pt}
\setlength{\parskip}{0.5em}

% Section formatting
\usepackage{titlesec}
\titleformat{\section}
  {\Large\bfseries\color{mlpurple}}
  {\thesection}{1em}{}
\titleformat{\subsection}
  {\large\bfseries\color{mlblue}}
  {\thesubsection}{1em}{}
\titleformat{\subsubsection}
  {\normalsize\bfseries}
  {\thesubsubsection}{1em}{}

% Lists
\usepackage{enumitem}
\setlist[itemize]{topsep=0.5em, itemsep=0.25em}
\setlist[enumerate]{topsep=0.5em, itemsep=0.25em}

% Tight list support for pandoc
\providecommand{\tightlist}{%
  \setlength{\itemsep}{0pt}\setlength{\parskip}{0pt}}

% Title
\title{\Huge\bfseries\color{mlpurple} Finance Applications of ML - Basic Handout}
\author{Machine Learning for Smarter Innovation}
\date{}

\begin{document}

\maketitle
\thispagestyle{fancy}

\section{Finance Applications of ML - Basic
Handout}\label{finance-applications-of-ml---basic-handout}

\textbf{Target Audience}: Finance professionals with no ML background
\textbf{Duration}: 30 minutes reading \textbf{Level}: Basic (no math,
practical focus)

\begin{center}\rule{0.5\linewidth}{0.5pt}\end{center}

\subsection{ML in Finance: Overview}\label{ml-in-finance-overview}

Machine learning transforms finance by: - Automating decisions that
humans make slowly - Finding patterns in massive datasets - Making
predictions based on historical data - Managing risk more precisely

\begin{center}\rule{0.5\linewidth}{0.5pt}\end{center}

\subsection{Key Applications}\label{key-applications}

\subsubsection{1. Credit Scoring}\label{credit-scoring}

\textbf{Problem}: Decide who gets a loan \textbf{ML Solution}: Predict
default probability from application data

\textbf{Benefits}: - Faster decisions (seconds vs days) - More
consistent than human judgment - Can process more data points - Reduces
discrimination when done right

\textbf{Example Features}: Income, employment history, existing debts,
payment history

\subsubsection{2. Fraud Detection}\label{fraud-detection}

\textbf{Problem}: Find suspicious transactions \textbf{ML Solution}:
Flag anomalies that deviate from normal patterns

\textbf{Benefits}: - Real-time detection - Adapts to new fraud patterns
- Reduces false positives over time - Scales to millions of transactions

\textbf{Example}: Credit card company blocks transaction in foreign
country you've never visited

\subsubsection{3. Algorithmic Trading}\label{algorithmic-trading}

\textbf{Problem}: Execute trades optimally \textbf{ML Solution}: Predict
price movements, optimize execution

\textbf{Applications}: - High-frequency trading (milliseconds) -
Portfolio rebalancing - Market making - Sentiment-based trading

\textbf{Note}: Most retail investors shouldn't compete here -
institutions have advantages

\subsubsection{4. Portfolio Management}\label{portfolio-management}

\textbf{Problem}: Allocate assets to maximize returns for given risk
\textbf{ML Solution}: Optimize portfolios using predicted returns and
correlations

\textbf{Robo-Advisors}: Automated portfolio management (Betterment,
Wealthfront) - Low fees - Tax-loss harvesting - Automatic rebalancing

\subsubsection{5. Risk Management}\label{risk-management}

\textbf{Problem}: Quantify potential losses \textbf{ML Solution}: Better
estimate of Value at Risk (VaR) and stress scenarios

\textbf{Applications}: - Market risk (price changes) - Credit risk
(defaults) - Operational risk (fraud, errors) - Liquidity risk (can't
sell assets)

\begin{center}\rule{0.5\linewidth}{0.5pt}\end{center}

\subsection{Key Concepts}\label{key-concepts}

\subsubsection{Value at Risk (VaR)}\label{value-at-risk-var}

``What's the maximum I could lose in a bad day?''

\textbf{Example}: ``95\% VaR of \$1M means there's a 5\% chance of
losing more than \$1M''

\subsubsection{Portfolio Optimization}\label{portfolio-optimization}

``How do I balance risk and return?''

\textbf{Key Idea}: Diversification - don't put all eggs in one basket

\subsubsection{Backtesting}\label{backtesting}

``Would this strategy have worked in the past?''

\textbf{Warning}: Past performance doesn't guarantee future results.
Overfitting is a major risk.

\begin{center}\rule{0.5\linewidth}{0.5pt}\end{center}

\subsection{Regulatory Requirements}\label{regulatory-requirements}

\subsubsection{SR 11-7 (Federal Reserve)}\label{sr-11-7-federal-reserve}

\begin{itemize}
\tightlist
\item
  Model risk management
\item
  Independent validation
\item
  Ongoing monitoring
\item
  Documentation requirements
\end{itemize}

\subsubsection{MiFID II (Europe)}\label{mifid-ii-europe}

\begin{itemize}
\tightlist
\item
  Algorithmic trading controls
\item
  Best execution requirements
\item
  Transparency rules
\end{itemize}

\subsubsection{Basel III}\label{basel-iii}

\begin{itemize}
\tightlist
\item
  Capital requirements
\item
  Risk-weighted assets
\item
  Stress testing
\end{itemize}

\textbf{Key Point}: Models in finance are heavily regulated. You can't
just deploy any ML model.

\begin{center}\rule{0.5\linewidth}{0.5pt}\end{center}

\subsection{When ML Works in Finance}\label{when-ml-works-in-finance}

\subsubsection{Good Fit:}\label{good-fit}

\begin{itemize}
\tightlist
\item
  Large historical datasets available
\item
  Patterns are relatively stable
\item
  Decisions are frequent and similar
\item
  Speed matters
\item
  Human bias is a concern
\end{itemize}

\subsubsection{Poor Fit:}\label{poor-fit}

\begin{itemize}
\tightlist
\item
  Unprecedented events (black swans)
\item
  Data is sparse or unreliable
\item
  Regulations require human judgment
\item
  Explainability is critical
\item
  Market conditions change fundamentally
\end{itemize}

\begin{center}\rule{0.5\linewidth}{0.5pt}\end{center}

\subsection{Common Pitfalls}\label{common-pitfalls}

\subsubsection{1. Overfitting to Historical
Data}\label{overfitting-to-historical-data}

\begin{itemize}
\tightlist
\item
  Strategy worked in backtest but fails live
\item
  Solution: Out-of-sample testing, walk-forward validation
\end{itemize}

\subsubsection{2. Look-Ahead Bias}\label{look-ahead-bias}

\begin{itemize}
\tightlist
\item
  Using information that wouldn't have been available
\item
  Solution: Strict temporal separation
\end{itemize}

\subsubsection{3. Survivorship Bias}\label{survivorship-bias}

\begin{itemize}
\tightlist
\item
  Only analyzing companies that still exist
\item
  Solution: Include delisted/bankrupt companies
\end{itemize}

\subsubsection{4. Data Snooping}\label{data-snooping}

\begin{itemize}
\tightlist
\item
  Testing many strategies, keeping only winners
\item
  Solution: Pre-register hypotheses, adjust for multiple testing
\end{itemize}

\subsubsection{5. Ignoring Transaction
Costs}\label{ignoring-transaction-costs}

\begin{itemize}
\tightlist
\item
  Strategy profitable on paper, loses money in practice
\item
  Solution: Include realistic costs in backtests
\end{itemize}

\begin{center}\rule{0.5\linewidth}{0.5pt}\end{center}

\subsection{Getting Started Checklist}\label{getting-started-checklist}

\subsubsection{For Credit/Risk Models:}\label{for-creditrisk-models}

\begin{itemize}
\tightlist
\item[$\square$]
  Understand regulatory requirements
\item[$\square$]
  Document data sources and preprocessing
\item[$\square$]
  Establish baseline (simple model)
\item[$\square$]
  Plan for model monitoring
\item[$\square$]
  Prepare explainability reports
\end{itemize}

\subsubsection{For Trading Strategies:}\label{for-trading-strategies}

\begin{itemize}
\tightlist
\item[$\square$]
  Use realistic backtesting
\item[$\square$]
  Account for transaction costs
\item[$\square$]
  Test on out-of-sample data
\item[$\square$]
  Start with paper trading
\item[$\square$]
  Size positions conservatively
\end{itemize}

\begin{center}\rule{0.5\linewidth}{0.5pt}\end{center}

\subsection{Tools and Platforms}\label{tools-and-platforms}

\subsubsection{Python Libraries:}\label{python-libraries}

\begin{itemize}
\tightlist
\item
  \textbf{pandas}: Data manipulation
\item
  \textbf{numpy}: Numerical computing
\item
  \textbf{scikit-learn}: General ML
\item
  \textbf{statsmodels}: Statistical models
\item
  \textbf{QuantLib}: Derivatives pricing
\end{itemize}

\subsubsection{Platforms:}\label{platforms}

\begin{itemize}
\tightlist
\item
  \textbf{Bloomberg Terminal}: Market data, analytics
\item
  \textbf{Refinitiv}: Financial data
\item
  \textbf{Alpaca}: Algorithmic trading API
\item
  \textbf{QuantConnect}: Backtesting platform
\end{itemize}

\begin{center}\rule{0.5\linewidth}{0.5pt}\end{center}

\subsection{Key Terms}\label{key-terms}

\begin{longtable}[]{@{}ll@{}}
\toprule\noalign{}
Term & Definition \\
\midrule\noalign{}
\endhead
\bottomrule\noalign{}
\endlastfoot
VaR & Maximum expected loss at confidence level \\
Sharpe Ratio & Risk-adjusted return measure \\
Alpha & Excess return above benchmark \\
Beta & Sensitivity to market movements \\
Drawdown & Peak-to-trough decline \\
Backtesting & Testing strategy on historical data \\
\end{longtable}

\begin{center}\rule{0.5\linewidth}{0.5pt}\end{center}

\subsection{Ethics in Finance ML}\label{ethics-in-finance-ml}

\subsubsection{Fair Lending:}\label{fair-lending}

\begin{itemize}
\tightlist
\item
  Models must not discriminate on protected characteristics
\item
  Even indirect discrimination (proxy variables) is problematic
\item
  Regular fairness audits required
\end{itemize}

\subsubsection{Market Manipulation:}\label{market-manipulation}

\begin{itemize}
\tightlist
\item
  Some algorithmic strategies may be illegal
\item
  Spoofing, layering, and front-running are prohibited
\item
  Ensure compliance with market rules
\end{itemize}

\subsubsection{Systemic Risk:}\label{systemic-risk}

\begin{itemize}
\tightlist
\item
  Many algorithms using similar strategies can amplify crashes
\item
  Flash crashes have occurred
\item
  Consider market impact
\end{itemize}

\begin{center}\rule{0.5\linewidth}{0.5pt}\end{center}

\subsection{Next Steps}\label{next-steps}

\begin{enumerate}
\def\labelenumi{\arabic{enumi}.}
\tightlist
\item
  \textbf{Learn}: Take a quantitative finance course
\item
  \textbf{Practice}: Use paper trading to test ideas
\item
  \textbf{Read}: Follow financial ML research
\item
  \textbf{Comply}: Understand regulatory requirements
\item
  \textbf{Proceed}: Read intermediate handout for implementation
\end{enumerate}

\begin{center}\rule{0.5\linewidth}{0.5pt}\end{center}

\emph{In finance, the stakes are real money. Always validate thoroughly,
comply with regulations, and remember that models can fail.}

\end{document}
