% Pandoc LaTeX template for course handouts
% Usage: pandoc input.md -o output.tex --template=handout_template.tex --listings
\documentclass[10pt,a4paper]{article}

% Geometry
\usepackage[margin=2.5cm]{geometry}

% Fonts and encoding
\usepackage[T1]{fontenc}
\usepackage[utf8]{inputenc}
\usepackage{lmodern}

% Math
\usepackage{amsmath,amssymb,amsfonts}

% Tables
\usepackage{booktabs}
\usepackage{longtable}
\usepackage{array}

% Graphics
\usepackage{graphicx}

% Colors
\usepackage{xcolor}
\definecolor{mlpurple}{RGB}{51,51,178}
\definecolor{mlblue}{RGB}{0,102,204}
\definecolor{codebg}{RGB}{248,248,248}
\definecolor{codeframe}{RGB}{200,200,200}

% Code listings
\usepackage{listings}
\lstset{
  backgroundcolor=\color{codebg},
  basicstyle=\ttfamily\small,
  breakatwhitespace=false,
  breaklines=true,
  captionpos=b,
  commentstyle=\color{gray},
  frame=single,
  framerule=0.5pt,
  rulecolor=\color{codeframe},
  keepspaces=true,
  keywordstyle=\color{mlblue}\bfseries,
  language=Python,
  numbers=none,
  showspaces=false,
  showstringspaces=false,
  showtabs=false,
  stringstyle=\color{mlpurple},
  tabsize=4,
  xleftmargin=0.5em,
  xrightmargin=0.5em,
  aboveskip=1em,
  belowskip=1em
}

% Hyperlinks
\usepackage{hyperref}
\hypersetup{
  colorlinks=true,
  linkcolor=mlblue,
  urlcolor=mlblue,
  citecolor=mlblue
}

% Headers and footers
\usepackage{fancyhdr}
\pagestyle{fancy}
\fancyhf{}
\lhead{\textcolor{mlpurple}{\textbf{Supervised Learning}}}
\rhead{\textcolor{gray}{Intermediate}}
\cfoot{\thepage}
\renewcommand{\headrulewidth}{0.4pt}
\renewcommand{\footrulewidth}{0pt}

% Paragraph spacing
\setlength{\parindent}{0pt}
\setlength{\parskip}{0.5em}

% Section formatting
\usepackage{titlesec}
\titleformat{\section}
  {\Large\bfseries\color{mlpurple}}
  {\thesection}{1em}{}
\titleformat{\subsection}
  {\large\bfseries\color{mlblue}}
  {\thesubsection}{1em}{}
\titleformat{\subsubsection}
  {\normalsize\bfseries}
  {\thesubsubsection}{1em}{}

% Lists
\usepackage{enumitem}
\setlist[itemize]{topsep=0.5em, itemsep=0.25em}
\setlist[enumerate]{topsep=0.5em, itemsep=0.25em}

% Tight list support for pandoc
\providecommand{\tightlist}{%
  \setlength{\itemsep}{0pt}\setlength{\parskip}{0pt}}

% Title
\title{\Huge\bfseries\color{mlpurple} Week 00b Intermediate: Supervised Learning Implementation}
\author{Machine Learning for Smarter Innovation}
\date{}

\begin{document}

\maketitle
\thispagestyle{fancy}

\section{Week 00b Intermediate: Supervised Learning
Implementation}\label{week-00b-intermediate-supervised-learning-implementation}

\subsection{Regression Example: House Price
Prediction}\label{regression-example-house-price-prediction}

\begin{lstlisting}[language=Python]
from sklearn.linear_model import LinearRegression, Ridge, Lasso
from sklearn.ensemble import RandomForestRegressor, GradientBoostingRegressor
from sklearn.metrics import mean_squared_error, r2_score
import numpy as np
import pandas as pd

# Load data (example)
from sklearn.datasets import fetch_california_housing
housing = fetch_california_housing()
X, y = housing.data, housing.target

# Split
from sklearn.model_selection import train_test_split
X_train, X_test, y_train, y_test = train_test_split(X, y, test_size=0.2, random_state=42)

# Compare models
models = {
    'Linear': LinearRegression(),
    'Ridge': Ridge(alpha=1.0),
    'Lasso': Lasso(alpha=0.1),
    'RandomForest': RandomForestRegressor(n_estimators=100),
    'GradientBoosting': GradientBoostingRegressor(n_estimators=100)
}

for name, model in models.items():
    model.fit(X_train, y_train)
    y_pred = model.predict(X_test)
    rmse = np.sqrt(mean_squared_error(y_test, y_pred))
    r2 = r2_score(y_test, y_pred)
    print(f"{name}: RMSE={rmse:.3f}, R²={r2:.3f}")
\end{lstlisting}

\subsection{Classification: Binary and
Multi-class}\label{classification-binary-and-multi-class}

\begin{lstlisting}[language=Python]
from sklearn.datasets import load_breast_cancer, load_iris
from sklearn.linear_model import LogisticRegression
from sklearn.tree import DecisionTreeClassifier
from sklearn.ensemble import RandomForestClassifier
from sklearn.svm import SVC
from sklearn.metrics import classification_report, confusion_matrix

# Binary classification
cancer = load_breast_cancer()
X, y = cancer.data, cancer.target
X_train, X_test, y_train, y_test = train_test_split(X, y, test_size=0.2)

clf = LogisticRegression(max_iter=10000)
clf.fit(X_train, y_train)
y_pred = clf.predict(X_test)

print(classification_report(y_test, y_pred, target_names=cancer.target_names))
\end{lstlisting}

\subsection{Hyperparameter Tuning}\label{hyperparameter-tuning}

\begin{lstlisting}[language=Python]
from sklearn.model_selection import GridSearchCV, RandomizedSearchCV

param_grid = {
    'n_estimators': [50, 100, 200],
    'max_depth': [5, 10, 20, None],
    'min_samples_split': [2, 5, 10]
}

rf = RandomForestClassifier()
grid = GridSearchCV(rf, param_grid, cv=5, scoring='accuracy', n_jobs=-1)
grid.fit(X_train, y_train)

print(f"Best params: {grid.best_params_}")
print(f"Best CV score: {grid.best_score_:.3f}")
\end{lstlisting}

\subsection{Feature Engineering}\label{feature-engineering}

\begin{lstlisting}[language=Python]
from sklearn.preprocessing import PolynomialFeatures, StandardScaler
from sklearn.pipeline import Pipeline

# Polynomial features
pipeline = Pipeline([
    ('poly', PolynomialFeatures(degree=2)),
    ('scaler', StandardScaler()),
    ('model', Ridge(alpha=1.0))
])

pipeline.fit(X_train, y_train)
score = pipeline.score(X_test, y_test)
\end{lstlisting}

\subsection{Practice Exercises}\label{practice-exercises}

\begin{enumerate}
\def\labelenumi{\arabic{enumi}.}
\tightlist
\item
  Load Titanic dataset, predict survival
\item
  Compare logistic vs SVM on iris dataset
\item
  Build ensemble of 3 different models
\item
  Tune XGBoost hyperparameters with RandomizedSearchCV
\end{enumerate}

\end{document}
