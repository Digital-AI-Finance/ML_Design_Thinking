% Pandoc LaTeX template for course handouts
% Usage: pandoc input.md -o output.tex --template=handout_template.tex --listings
\documentclass[10pt,a4paper]{article}

% Geometry
\usepackage[margin=2.5cm]{geometry}

% Fonts and encoding
\usepackage[T1]{fontenc}
\usepackage[utf8]{inputenc}
\usepackage{lmodern}

% Math
\usepackage{amsmath,amssymb,amsfonts}

% Tables
\usepackage{booktabs}
\usepackage{longtable}
\usepackage{array}

% Graphics
\usepackage{graphicx}

% Colors
\usepackage{xcolor}
\definecolor{mlpurple}{RGB}{51,51,178}
\definecolor{mlblue}{RGB}{0,102,204}
\definecolor{codebg}{RGB}{248,248,248}
\definecolor{codeframe}{RGB}{200,200,200}

% Code listings
\usepackage{listings}
\lstset{
  backgroundcolor=\color{codebg},
  basicstyle=\ttfamily\small,
  breakatwhitespace=false,
  breaklines=true,
  captionpos=b,
  commentstyle=\color{gray},
  frame=single,
  framerule=0.5pt,
  rulecolor=\color{codeframe},
  keepspaces=true,
  keywordstyle=\color{mlblue}\bfseries,
  language=Python,
  numbers=none,
  showspaces=false,
  showstringspaces=false,
  showtabs=false,
  stringstyle=\color{mlpurple},
  tabsize=4,
  xleftmargin=0.5em,
  xrightmargin=0.5em,
  aboveskip=1em,
  belowskip=1em
}

% Hyperlinks
\usepackage{hyperref}
\hypersetup{
  colorlinks=true,
  linkcolor=mlblue,
  urlcolor=mlblue,
  citecolor=mlblue
}

% Headers and footers
\usepackage{fancyhdr}
\pagestyle{fancy}
\fancyhf{}
\lhead{\textcolor{mlpurple}{\textbf{Structured Output}}}
\rhead{\textcolor{gray}{Basic}}
\cfoot{\thepage}
\renewcommand{\headrulewidth}{0.4pt}
\renewcommand{\footrulewidth}{0pt}

% Paragraph spacing
\setlength{\parindent}{0pt}
\setlength{\parskip}{0.5em}

% Section formatting
\usepackage{titlesec}
\titleformat{\section}
  {\Large\bfseries\color{mlpurple}}
  {\thesection}{1em}{}
\titleformat{\subsection}
  {\large\bfseries\color{mlblue}}
  {\thesubsection}{1em}{}
\titleformat{\subsubsection}
  {\normalsize\bfseries}
  {\thesubsubsection}{1em}{}

% Lists
\usepackage{enumitem}
\setlist[itemize]{topsep=0.5em, itemsep=0.25em}
\setlist[enumerate]{topsep=0.5em, itemsep=0.25em}

% Tight list support for pandoc
\providecommand{\tightlist}{%
  \setlength{\itemsep}{0pt}\setlength{\parskip}{0pt}}

% Title
\title{\Huge\bfseries\color{mlpurple} Handout 1: Getting Reliable AI Outputs}
\author{Machine Learning for Smarter Innovation}
\date{}

\begin{document}

\maketitle
\thispagestyle{fancy}

\section{Handout 1: Getting Reliable AI
Outputs}\label{handout-1-getting-reliable-ai-outputs}

\subsection{A Beginner's Guide to Structured
Data}\label{a-beginners-guide-to-structured-data}

\subsubsection{What's the Problem?}\label{whats-the-problem}

When you ask AI to extract information, it sometimes gives you answers
in random formats: - Sometimes it says ``5 stars'', sometimes ``five out
of five'', sometimes just ``excellent'' - You can't use this in a
database or spreadsheet - You have to manually fix every response - It's
unreliable for real applications

\subsubsection{What's the Solution?}\label{whats-the-solution}

\textbf{Structured outputs} - asking AI to give you data in a specific,
predictable format (like filling out a form).

\subsubsection{Real Example}\label{real-example}

\textbf{Unstructured (Bad):}

\begin{lstlisting}
The restaurant was great! I'd give it 5 stars.
Food was amazing, service excellent. Price was around $30 per person.
\end{lstlisting}

Problem: How do you extract the rating? Is it 5 or 5.0? Where's the
price? What format?

\textbf{Structured (Good):}

\begin{lstlisting}
{
  "rating": 5,
  "food_quality": 5,
  "service_quality": 5,
  "price_per_person": 30,
  "price_level": "moderate"
}
\end{lstlisting}

Solution: Every field is clear, in the right format, ready to use!

\subsubsection{Why This Matters}\label{why-this-matters}

\begin{itemize}
\tightlist
\item
  \textbf{Databases need consistent formats} - You can't mix ``5 stars''
  and ``five''
\item
  \textbf{Automation breaks} - Random formats make automation impossible
\item
  \textbf{Trust} - Consistent outputs = reliable system
\end{itemize}

\subsubsection{Key Concept: JSON Schema}\label{key-concept-json-schema}

Think of it like a form template that AI must fill out correctly.

\textbf{Your template says:} - rating must be a number between 1 and 5 -
price\_level must be either ``cheap'', ``moderate'', or ``expensive'' -
service\_quality is required

\textbf{AI must follow these rules} or the output is rejected.

\subsubsection{How to Get Structured
Outputs}\label{how-to-get-structured-outputs}

\paragraph{Option 1: Use ChatGPT with Instructions (No
Coding)}\label{option-1-use-chatgpt-with-instructions-no-coding}

Instead of: \textgreater{} ``Analyze this restaurant review''

Try: \textgreater{} ``Extract this information in JSON format:
\textgreater{} \{ \textgreater{}''rating'': (number 1-5), \textgreater{}
``price\_level'': (cheap/moderate/expensive), \textgreater{}
``food\_quality'': (number 1-5) \textgreater{} \}''

Better!

\paragraph{Option 2: Better Prompts}\label{option-2-better-prompts}

\textbf{Basic Prompt (70\% success):} ``Extract the rating from this
review''

\textbf{Role-Based Prompt (80\% success):} ``You are a data extraction
expert. Extract the numerical rating (1-5) from this review.''

\textbf{Step-by-Step Prompt (90\% success):} ``Step 1: Read the review
Step 2: Find mentions of quality or rating Step 3: Convert to a number
from 1-5 Step 4: Return just the number''

More specific = more reliable!

\subsubsection{Temperature Setting}\label{temperature-setting}

Temperature controls creativity vs consistency:

\begin{itemize}
\tightlist
\item
  \textbf{Temperature 0} → Same answer every time (use for data
  extraction)
\item
  \textbf{Temperature 0.7} → Creative but less consistent (use for
  writing)
\item
  \textbf{Temperature 1.5} → Very creative, very different (use for
  brainstorming)
\end{itemize}

\textbf{Rule of Thumb:} For structured data, use temperature 0-0.3

\subsubsection{When Do You Need Structured
Outputs?}\label{when-do-you-need-structured-outputs}

\paragraph{Use Structured:}\label{use-structured}

\begin{itemize}
\tightlist
\item
  Filling out forms
\item
  Extracting data from documents
\item
  Building databases
\item
  Automated workflows
\item
  API integrations
\item
  Anything that needs consistency
\end{itemize}

\paragraph{Use Unstructured (Regular
Text):}\label{use-unstructured-regular-text}

\begin{itemize}
\tightlist
\item
  Creative writing
\item
  Explanations
\item
  Conversations
\item
  Brainstorming
\item
  Marketing copy
\end{itemize}

\subsubsection{Checklist for
Reliability}\label{checklist-for-reliability}

Before launching your AI system:

\begin{itemize}
\tightlist
\item[$\square$]
  Defined exactly what format you need
\item[$\square$]
  Wrote clear instructions for the AI
\item[$\square$]
  Tested with 10+ examples
\item[$\square$]
  Set temperature to 0-0.3
\item[$\square$]
  Have a backup plan if AI fails
\item[$\square$]
  Tested edge cases (weird inputs)
\item[$\square$]
  Someone else reviewed your system
\end{itemize}

\subsubsection{Common Mistakes}\label{common-mistakes}

\begin{enumerate}
\def\labelenumi{\arabic{enumi}.}
\tightlist
\item
  \textbf{``The AI understands what I want''}

  \begin{itemize}
  \tightlist
  \item
    No! Be specific. Show exact format wanted.
  \end{itemize}
\item
  \textbf{``I'll parse the text later''}

  \begin{itemize}
  \tightlist
  \item
    No! Get structured output directly. Parsing is error-prone.
  \end{itemize}
\item
  \textbf{``It worked once, ship it!''}

  \begin{itemize}
  \tightlist
  \item
    No! Test with 50-100 examples. One success means nothing.
  \end{itemize}
\item
  \textbf{``Users will understand errors''}

  \begin{itemize}
  \tightlist
  \item
    No! Add friendly error messages, not technical jargon.
  \end{itemize}
\item
  \textbf{``AI never makes mistakes''}

  \begin{itemize}
  \tightlist
  \item
    No! Always have human review for important decisions.
  \end{itemize}
\end{enumerate}

\subsubsection{Quick Wins}\label{quick-wins}

\paragraph{Win 1: Add Examples}\label{win-1-add-examples}

Show AI 2-3 examples of the format you want. Success rate jumps 15-20\%.

\paragraph{Win 2: Break Down Steps}\label{win-2-break-down-steps}

Instead of ``extract everything'', do: 1. Extract rating 2. Extract
price 3. Extract categories

One thing at a time = more reliable.

\paragraph{Win 3: Validate Results}\label{win-3-validate-results}

Check if the output makes sense: - Is rating between 1-5? - Is price a
positive number? - Are all required fields present?

Reject bad outputs, don't use them!

\subsubsection{What Success Looks Like}\label{what-success-looks-like}

\begin{itemize}
\tightlist
\item
  \textbf{90\%+ of outputs} are correct without human review
\item
  \textbf{5\%} need minor corrections
\item
  \textbf{5\%} fail completely and need manual entry
\end{itemize}

This is normal! No AI system is 100\% perfect.

\subsubsection{Red Flags to Watch For}\label{red-flags-to-watch-for}

Stop and fix if you see: - Success rate below 80\% - Inconsistent field
formats - Frequent complete failures - Users complaining about errors -
Manual work isn't decreasing

\subsubsection{Next Steps}\label{next-steps}

\begin{enumerate}
\def\labelenumi{\arabic{enumi}.}
\tightlist
\item
  \textbf{Try it yourself} - Use ChatGPT with structured prompts
\item
  \textbf{Start simple} - One field at a time
\item
  \textbf{Test thoroughly} - 50+ examples before trusting it
\item
  \textbf{Get feedback} - Show to colleagues
\item
  \textbf{Improve gradually} - Add complexity slowly
\end{enumerate}

\subsubsection{Real-World Example: Invoice
Processing}\label{real-world-example-invoice-processing}

\textbf{Before (Unstructured):} - 3 hours per invoice to manually enter
data - 3\% error rate from typos - Cannot scale

\textbf{After (Structured):} - 2 minutes per invoice (AI extracts, human
verifies) - 0.2\% error rate - Can handle 100x volume

\textbf{Key:} AI extracts to structured format, human just checks and
fixes.

\subsubsection{Resources for Beginners}\label{resources-for-beginners}

\begin{enumerate}
\def\labelenumi{\arabic{enumi}.}
\tightlist
\item
  \textbf{ChatGPT Playground} - Free, try structured prompts
\item
  \textbf{This course handouts} - Read intermediate handout next
\item
  \textbf{JSON formatter} - jsonformatter.org (see what JSON looks like)
\item
  \textbf{Practice dataset} - Restaurant reviews (ask instructor)
\end{enumerate}

\subsubsection{Remember}\label{remember}

\begin{itemize}
\tightlist
\item
  \textbf{Structure beats creativity for production}
\item
  \textbf{Be specific in your requests}
\item
  \textbf{Test, test, test}
\item
  \textbf{Users should always be able to override AI}
\item
  \textbf{Start simple, add complexity gradually}
\end{itemize}

\subsubsection{Key Takeaway}\label{key-takeaway}

Getting reliable AI outputs is about being specific, using the right
format (structured/JSON), and thorough testing. It's not magic - it's
careful engineering!

\begin{center}\rule{0.5\linewidth}{0.5pt}\end{center}

\emph{Next: Read Handout 2 for code examples and implementation}

\end{document}
