% Pandoc LaTeX template for course handouts
% Usage: pandoc input.md -o output.tex --template=handout_template.tex --listings
\documentclass[10pt,a4paper]{article}

% Geometry
\usepackage[margin=2.5cm]{geometry}

% Fonts and encoding
\usepackage[T1]{fontenc}
\usepackage[utf8]{inputenc}
\usepackage{lmodern}

% Math
\usepackage{amsmath,amssymb,amsfonts}

% Tables
\usepackage{booktabs}
\usepackage{longtable}
\usepackage{array}

% Graphics
\usepackage{graphicx}

% Colors
\usepackage{xcolor}
\definecolor{mlpurple}{RGB}{51,51,178}
\definecolor{mlblue}{RGB}{0,102,204}
\definecolor{codebg}{RGB}{248,248,248}
\definecolor{codeframe}{RGB}{200,200,200}

% Code listings
\usepackage{listings}
\lstset{
  backgroundcolor=\color{codebg},
  basicstyle=\ttfamily\small,
  breakatwhitespace=false,
  breaklines=true,
  captionpos=b,
  commentstyle=\color{gray},
  frame=single,
  framerule=0.5pt,
  rulecolor=\color{codeframe},
  keepspaces=true,
  keywordstyle=\color{mlblue}\bfseries,
  language=Python,
  numbers=none,
  showspaces=false,
  showstringspaces=false,
  showtabs=false,
  stringstyle=\color{mlpurple},
  tabsize=4,
  xleftmargin=0.5em,
  xrightmargin=0.5em,
  aboveskip=1em,
  belowskip=1em
}

% Hyperlinks
\usepackage{hyperref}
\hypersetup{
  colorlinks=true,
  linkcolor=mlblue,
  urlcolor=mlblue,
  citecolor=mlblue
}

% Headers and footers
\usepackage{fancyhdr}
\pagestyle{fancy}
\fancyhf{}
\lhead{\textcolor{mlpurple}{\textbf{Generative AI}}}
\rhead{\textcolor{gray}{Basic}}
\cfoot{\thepage}
\renewcommand{\headrulewidth}{0.4pt}
\renewcommand{\footrulewidth}{0pt}

% Paragraph spacing
\setlength{\parindent}{0pt}
\setlength{\parskip}{0.5em}

% Section formatting
\usepackage{titlesec}
\titleformat{\section}
  {\Large\bfseries\color{mlpurple}}
  {\thesection}{1em}{}
\titleformat{\subsection}
  {\large\bfseries\color{mlblue}}
  {\thesubsection}{1em}{}
\titleformat{\subsubsection}
  {\normalsize\bfseries}
  {\thesubsubsection}{1em}{}

% Lists
\usepackage{enumitem}
\setlist[itemize]{topsep=0.5em, itemsep=0.25em}
\setlist[enumerate]{topsep=0.5em, itemsep=0.25em}

% Tight list support for pandoc
\providecommand{\tightlist}{%
  \setlength{\itemsep}{0pt}\setlength{\parskip}{0pt}}

% Title
\title{\Huge\bfseries\color{mlpurple} Handout 1: Getting Started with AI Prototyping (Basic Level)}
\author{Machine Learning for Smarter Innovation}
\date{}

\begin{document}

\maketitle
\thispagestyle{fancy}

\section{Handout 1: Getting Started with AI Prototyping (Basic
Level)}\label{handout-1-getting-started-with-ai-prototyping-basic-level}

\subsection{What is Generative AI?}\label{what-is-generative-ai}

Generative AI creates new content - text, images, code, audio - based on
patterns learned from existing data. Think of it as a creative assistant
that can help you prototype ideas faster.

\subsubsection{Common Generative AI
Tools}\label{common-generative-ai-tools}

\begin{enumerate}
\def\labelenumi{\arabic{enumi}.}
\tightlist
\item
  \textbf{ChatGPT/Claude} - Text generation, brainstorming, writing
\item
  \textbf{DALL-E/Midjourney} - Image creation from text descriptions
\item
  \textbf{GitHub Copilot} - Code writing assistant
\item
  \textbf{Canva AI} - Design automation
\end{enumerate}

\subsection{Your First AI Prototype:
Step-by-Step}\label{your-first-ai-prototype-step-by-step}

\subsubsection{Step 1: Define Your Goal}\label{step-1-define-your-goal}

\begin{itemize}
\tightlist
\item
  What problem are you solving?
\item
  Who is your target user?
\item
  What would success look like?
\end{itemize}

\subsubsection{Step 2: Choose Your Tool}\label{step-2-choose-your-tool}

\begin{itemize}
\tightlist
\item
  \textbf{For text/ideas}: ChatGPT (free tier available)
\item
  \textbf{For images}: Bing Image Creator (free with Microsoft account)
\item
  \textbf{For design}: Canva (free tier with AI features)
\end{itemize}

\subsubsection{Step 3: Write Your First
Prompt}\label{step-3-write-your-first-prompt}

\begin{lstlisting}
Bad Prompt: "Make a logo"

Good Prompt: "Create a minimalist logo for a sustainable
coffee shop called 'Green Bean' using earth tones and a
coffee plant leaf icon"
\end{lstlisting}

\subsubsection{Step 4: Iterate and
Refine}\label{step-4-iterate-and-refine}

\begin{itemize}
\tightlist
\item
  Generate 5-10 variations
\item
  Combine the best elements
\item
  Refine with more specific prompts
\end{itemize}

\subsection{Practical Exercise: 30-Minute Product
Concept}\label{practical-exercise-30-minute-product-concept}

\subsubsection{Materials Needed}\label{materials-needed}

\begin{itemize}
\tightlist
\item
  Computer with internet
\item
  Free ChatGPT account
\item
  Free Canva account
\end{itemize}

\subsubsection{Task}\label{task}

Create a mobile app concept for helping students manage study time.

\subsubsection{Process}\label{process}

\begin{enumerate}
\def\labelenumi{\arabic{enumi}.}
\tightlist
\item
  \textbf{Brainstorm (5 min)}

  \begin{itemize}
  \tightlist
  \item
    Ask ChatGPT: ``Generate 5 innovative features for a student time
    management app''
  \end{itemize}
\item
  \textbf{Develop Concept (10 min)}

  \begin{itemize}
  \tightlist
  \item
    Pick the best 3 features
  \item
    Ask for detailed descriptions
  \end{itemize}
\item
  \textbf{Create Visuals (10 min)}

  \begin{itemize}
  \tightlist
  \item
    Use Canva AI to generate app mockup
  \item
    Create simple logo
  \end{itemize}
\item
  \textbf{Write Description (5 min)}

  \begin{itemize}
  \tightlist
  \item
    Generate app store description with ChatGPT
  \end{itemize}
\end{enumerate}

\subsection{Tips for Success}\label{tips-for-success}

\subsubsection{DO:}\label{do}

\begin{itemize}
\tightlist
\item
  Start simple
\item
  Try multiple variations
\item
  Combine AI output with your ideas
\item
  Always review and edit AI content
\item
  Save your prompts for reuse
\end{itemize}

\subsubsection{DON'T:}\label{dont}

\begin{itemize}
\tightlist
\item
  Accept first output as final
\item
  Use AI content without review
\item
  Ignore ethical considerations
\item
  Share sensitive data
\end{itemize}

\subsection{Common Beginner Mistakes}\label{common-beginner-mistakes}

\begin{enumerate}
\def\labelenumi{\arabic{enumi}.}
\tightlist
\item
  \textbf{Vague Prompts} → Be specific!
\item
  \textbf{No Iteration} → Always generate multiple versions
\item
  \textbf{Blind Trust} → Always fact-check AI outputs
\item
  \textbf{Over-complexity} → Start with simple tasks
\end{enumerate}

\subsection{Free Resources to Explore}\label{free-resources-to-explore}

\begin{itemize}
\tightlist
\item
  \textbf{OpenAI Playground}: experiments.openai.com
\item
  \textbf{Hugging Face}: Free AI models to try
\item
  \textbf{Google Colab}: Free computing for AI experiments
\item
  \textbf{YouTube}: ``AI for beginners'' tutorials
\end{itemize}

\subsection{Ethics Reminder}\label{ethics-reminder}

\begin{itemize}
\tightlist
\item
  Always disclose when content is AI-generated
\item
  Don't use AI to create misleading content
\item
  Respect copyright and intellectual property
\item
  Consider bias in AI outputs
\end{itemize}

\subsection{Your Homework}\label{your-homework}

\begin{enumerate}
\def\labelenumi{\arabic{enumi}.}
\tightlist
\item
  Create 3 different product concepts using AI
\item
  Generate 5 variations of each
\item
  Document which prompts worked best
\item
  Share your favorite with the class
\end{enumerate}

\subsection{Questions to Consider}\label{questions-to-consider}

\begin{itemize}
\tightlist
\item
  How did AI speed up your creative process?
\item
  What limitations did you encounter?
\item
  How would you combine AI with traditional methods?
\item
  What ethical concerns arose?
\end{itemize}

\begin{center}\rule{0.5\linewidth}{0.5pt}\end{center}

\emph{Remember: AI is a tool to enhance your creativity, not replace it.
The best results come from human creativity guided by AI assistance.}

\end{document}
