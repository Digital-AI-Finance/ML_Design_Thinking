% Pandoc LaTeX template for course handouts
% Usage: pandoc input.md -o output.tex --template=handout_template.tex --listings
\documentclass[10pt,a4paper]{article}

% Geometry
\usepackage[margin=2.5cm]{geometry}

% Fonts and encoding
\usepackage[T1]{fontenc}
\usepackage[utf8]{inputenc}
\usepackage{lmodern}

% Math
\usepackage{amsmath,amssymb,amsfonts}

% Tables
\usepackage{booktabs}
\usepackage{longtable}
\usepackage{array}

% Graphics
\usepackage{graphicx}

% Colors
\usepackage{xcolor}
\definecolor{mlpurple}{RGB}{51,51,178}
\definecolor{mlblue}{RGB}{0,102,204}
\definecolor{codebg}{RGB}{248,248,248}
\definecolor{codeframe}{RGB}{200,200,200}

% Code listings
\usepackage{listings}
\lstset{
  backgroundcolor=\color{codebg},
  basicstyle=\ttfamily\small,
  breakatwhitespace=false,
  breaklines=true,
  captionpos=b,
  commentstyle=\color{gray},
  frame=single,
  framerule=0.5pt,
  rulecolor=\color{codeframe},
  keepspaces=true,
  keywordstyle=\color{mlblue}\bfseries,
  language=Python,
  numbers=none,
  showspaces=false,
  showstringspaces=false,
  showtabs=false,
  stringstyle=\color{mlpurple},
  tabsize=4,
  xleftmargin=0.5em,
  xrightmargin=0.5em,
  aboveskip=1em,
  belowskip=1em
}

% Hyperlinks
\usepackage{hyperref}
\hypersetup{
  colorlinks=true,
  linkcolor=mlblue,
  urlcolor=mlblue,
  citecolor=mlblue
}

% Headers and footers
\usepackage{fancyhdr}
\pagestyle{fancy}
\fancyhf{}
\lhead{\textcolor{mlpurple}{\textbf{NLP & Sentiment}}}
\rhead{\textcolor{gray}{Basic}}
\cfoot{\thepage}
\renewcommand{\headrulewidth}{0.4pt}
\renewcommand{\footrulewidth}{0pt}

% Paragraph spacing
\setlength{\parindent}{0pt}
\setlength{\parskip}{0.5em}

% Section formatting
\usepackage{titlesec}
\titleformat{\section}
  {\Large\bfseries\color{mlpurple}}
  {\thesection}{1em}{}
\titleformat{\subsection}
  {\large\bfseries\color{mlblue}}
  {\thesubsection}{1em}{}
\titleformat{\subsubsection}
  {\normalsize\bfseries}
  {\thesubsubsection}{1em}{}

% Lists
\usepackage{enumitem}
\setlist[itemize]{topsep=0.5em, itemsep=0.25em}
\setlist[enumerate]{topsep=0.5em, itemsep=0.25em}

% Tight list support for pandoc
\providecommand{\tightlist}{%
  \setlength{\itemsep}{0pt}\setlength{\parskip}{0pt}}

% Title
\title{\Huge\bfseries\color{mlpurple} Handout 1: Introduction to Sentiment Analysis (Basic Level)}
\author{Machine Learning for Smarter Innovation}
\date{}

\begin{document}

\maketitle
\thispagestyle{fancy}

\section{Handout 1: Introduction to Sentiment Analysis (Basic
Level)}\label{handout-1-introduction-to-sentiment-analysis-basic-level}

\subsection{Week 3 - NLP for Emotional
Context}\label{week-3---nlp-for-emotional-context}

\subsubsection{What is Sentiment
Analysis?}\label{what-is-sentiment-analysis}

Sentiment analysis is the computational study of people's opinions,
sentiments, emotions, and attitudes expressed in text. It helps us
understand how users feel about products, services, or experiences at
scale.

\subsubsection{Key Concepts}\label{key-concepts}

\paragraph{1. Sentiment Polarity}\label{sentiment-polarity}

\begin{itemize}
\tightlist
\item
  \textbf{Positive}: Expressing satisfaction, happiness, or approval
\item
  \textbf{Negative}: Expressing dissatisfaction, frustration, or
  disapproval\\
\item
  \textbf{Neutral}: Factual statements without clear emotion
\end{itemize}

\paragraph{2. Why Sentiment Analysis Matters for
Design}\label{why-sentiment-analysis-matters-for-design}

\begin{itemize}
\tightlist
\item
  Understand user pain points automatically
\item
  Identify delightful moments in the user journey
\item
  Prioritize features based on emotional impact
\item
  Monitor brand perception in real-time
\end{itemize}

\subsubsection{Traditional vs Machine Learning
Approaches}\label{traditional-vs-machine-learning-approaches}

\paragraph{Rule-Based Methods}\label{rule-based-methods}

\begin{lstlisting}[language=Python]
# Simple rule-based sentiment using word lists
positive_words = ['good', 'great', 'excellent', 'amazing', 'wonderful']
negative_words = ['bad', 'terrible', 'awful', 'horrible', 'poor']

def simple_sentiment(text):
    text_lower = text.lower()
    pos_count = sum(1 for word in positive_words if word in text_lower)
    neg_count = sum(1 for word in negative_words if word in text_lower)
    
    if pos_count > neg_count:
        return 'positive'
    elif neg_count > pos_count:
        return 'negative'
    else:
        return 'neutral'
\end{lstlisting}

\textbf{Limitations}: - Misses context (``not good'' would be positive)
- Can't detect sarcasm - Limited vocabulary - No understanding of word
relationships

\paragraph{Machine Learning Approach}\label{machine-learning-approach}

\begin{lstlisting}[language=Python]
from textblob import TextBlob

def ml_sentiment(text):
    analysis = TextBlob(text)
    # Polarity ranges from -1 (negative) to 1 (positive)
    if analysis.sentiment.polarity > 0.1:
        return 'positive'
    elif analysis.sentiment.polarity < -0.1:
        return 'negative'
    else:
        return 'neutral'
\end{lstlisting}

\subsubsection{Getting Started with
TextBlob}\label{getting-started-with-textblob}

\paragraph{Installation}\label{installation}

\begin{lstlisting}[language=bash]
pip install textblob
python -m textblob.download_corpora
\end{lstlisting}

\paragraph{Basic Usage}\label{basic-usage}

\begin{lstlisting}[language=Python]
from textblob import TextBlob

# Analyze a product review
review = "This product is amazing! The quality exceeded my expectations."
blob = TextBlob(review)

# Get sentiment
print(f"Polarity: {blob.sentiment.polarity}")  # 0.625 (positive)
print(f"Subjectivity: {blob.sentiment.subjectivity}")  # 0.6

# Polarity: -1 to 1 (negative to positive)
# Subjectivity: 0 to 1 (objective to subjective)
\end{lstlisting}

\subsubsection{Practice Exercise: Movie Review
Sentiment}\label{practice-exercise-movie-review-sentiment}

\paragraph{Task 1: Basic Analysis}\label{task-1-basic-analysis}

\begin{lstlisting}[language=Python]
reviews = [
    "This movie was fantastic! Best film of the year.",
    "Boring and predictable. Waste of time.",
    "The acting was okay but the plot was confusing.",
    "Absolutely loved it! Can't wait to watch again.",
    "Terrible movie. Do not recommend."
]

# Your task: Classify each review as positive, negative, or neutral
# Use both rule-based and TextBlob methods
# Compare the results
\end{lstlisting}

\paragraph{Task 2: Batch Processing}\label{task-2-batch-processing}

\begin{lstlisting}[language=Python]
def analyze_reviews(reviews):
    results = []
    for review in reviews:
        blob = TextBlob(review)
        sentiment = 'positive' if blob.sentiment.polarity > 0.1 else \
                   'negative' if blob.sentiment.polarity < -0.1 else 'neutral'
        results.append({
            'text': review,
            'sentiment': sentiment,
            'confidence': abs(blob.sentiment.polarity)
        })
    return results

# Analyze all reviews and find:
# 1. What percentage are positive?
# 2. Which review has the strongest sentiment?
# 3. Which reviews might need human review (low confidence)?
\end{lstlisting}

\subsubsection{Understanding Context
Matters}\label{understanding-context-matters}

Consider these examples: 1. ``This product is sick!'' → Context: Teen
slang (Positive) vs Medical (Negative) 2. ``It's fine\ldots{}'' →
Context: After complaint (Negative) vs First impression (Neutral) 3.
``Interesting choice'' → Context: Design critique (Negative) vs Genuine
interest (Positive)

\subsubsection{Common Pitfalls to Avoid}\label{common-pitfalls-to-avoid}

\begin{enumerate}
\def\labelenumi{\arabic{enumi}.}
\tightlist
\item
  \textbf{Ignoring Negation}: ``not bad'' ≠ ``bad''
\item
  \textbf{Missing Sarcasm}: ``Oh great, another bug!'' (Actually
  negative)
\item
  \textbf{Domain Specificity}: ``This app is the bomb!'' (Positive in
  casual, concerning in security)
\item
  \textbf{Mixed Sentiment}: ``Love the design but hate the price''
\end{enumerate}

\subsubsection{Real-World Applications}\label{real-world-applications}

\paragraph{Customer Reviews}\label{customer-reviews}

\begin{lstlisting}[language=Python]
# Analyzing product feedback
def analyze_product_reviews(reviews):
    sentiments = {'positive': 0, 'negative': 0, 'neutral': 0}
    issues = []
    
    for review in reviews:
        blob = TextBlob(review)
        
        # Classify sentiment
        if blob.sentiment.polarity > 0.1:
            sentiments['positive'] += 1
        elif blob.sentiment.polarity < -0.1:
            sentiments['negative'] += 1
            # Extract potential issues from negative reviews
            if 'bug' in review.lower() or 'crash' in review.lower():
                issues.append('Technical issues')
            elif 'expensive' in review.lower() or 'price' in review.lower():
                issues.append('Pricing concerns')
        else:
            sentiments['neutral'] += 1
    
    return sentiments, issues
\end{lstlisting}

\subsubsection{Key Takeaways}\label{key-takeaways}

\begin{enumerate}
\def\labelenumi{\arabic{enumi}.}
\tightlist
\item
  \textbf{Start Simple}: Begin with TextBlob for quick prototypes
\item
  \textbf{Context is King}: Always consider the domain and audience
\item
  \textbf{Validate Results}: Manual review of a sample is essential
\item
  \textbf{Iterate}: Improve based on false positives/negatives
\item
  \textbf{Combine Methods}: Use rules for obvious cases, ML for complex
  ones
\end{enumerate}

\subsubsection{Further Learning}\label{further-learning}

\begin{enumerate}
\def\labelenumi{\arabic{enumi}.}
\tightlist
\item
  Try analyzing your favorite product's reviews
\item
  Experiment with different sentiment thresholds
\item
  Build a simple sentiment dashboard
\item
  Explore emotion detection (beyond positive/negative)
\end{enumerate}

\subsubsection{Useful Resources}\label{useful-resources}

\begin{itemize}
\tightlist
\item
  TextBlob Documentation: https://textblob.readthedocs.io/
\item
  NLTK Sentiment Analysis: https://www.nltk.org/
\item
  Hugging Face Sentiment Models:
  https://huggingface.co/models?pipeline\_tag=sentiment-analysis
\item
  Real-time Twitter Sentiment: https://www.sentiment140.com/
\end{itemize}

\subsubsection{Assignment}\label{assignment}

Create a simple sentiment analyzer for a product of your choice: 1.
Collect 20-50 reviews (from Amazon, App Store, etc.) 2. Analyze
sentiment using TextBlob 3. Identify the top 3 positive and negative
themes 4. Create a simple visualization of results 5. Suggest one design
improvement based on negative sentiment

\textbf{Deliverable}: Python script + 1-page insights report

\begin{center}\rule{0.5\linewidth}{0.5pt}\end{center}

\emph{Remember: Sentiment analysis is a tool to understand users at
scale, but always validate insights with real user research!}

\end{document}
